\chapter{D'\Orba, la Terre}

\section{Les \Ea}

\Carac, sur ordre de \Cind, déposa le fragment de \Drisst en sa possession au cœur de la grande étoile \Naos. Sa lumière était belle et d'or. Son éclat ne repoussait pas l'obscurité comme les autres étoiles savaient le faire, les obscurités semblaient plutôt lui avoir cédé la place sans résister, par déférence envers une splendeur qu'elles ne pourraient jamais ravir. \Naos couva le fragment de \Drisst, et lentement celui se consuma. L'étoile palpita, de vives lumières s'élancèrent dans le monde et les \Ea, à travers la surface dorée de \Naos vinrent à leurs suites: \Oros d'abord, puis \Fercor, et enfin \Nio. En chacun d'eux brulait une part de \Mey et une part d'inconnu.

Il faut à présent aborder l'aspect des \Ea, car il peut changer dans les représentations qu'en ont faites les Hommes. Certains disent que nous en descendons directement, et que les \Ea sont nos lointains ancêtres. Ainsi, étant nos propres parents, nous ne pouvons que leur ressembler. D'autres pensent que nous avons surgi de leur imagination, et alors nous les voyons à notre propre image car c'est comme cela que nous avons pris l'habitude de les raconter.

Les rares Hommes devant lesquels ils sont apparus rapportent effectivement qu'ils avaient forme humaine, bien qu'elle soit différente de la notre, en stature notamment. Ils sont deux à trois fois plus grands que les plus grands Hommes d'aujourd'hui, et leur corps rayonne inlassablement une lumière. Parfois ils brillent d'une clarté pure, parfois ils s'encapent de lueurs enténébrées. De leur corps se dégage une chaleur intense mais ils peuvent aussi répandre autour d'eux le froid le plus glacial. Ils apparaissaient ainsi, mais nul n’aurait su a quoi ils ressemblaient lorsque leur image ne s’offrait à aucun regard. D'ailleurs les \Ea pouvaient prendre l'apparence qui leur convenaient, en fonction de ce qu’ils désiraient faire ou être aux yeux de leur interlocuteurs. Mais c'était avant qu'ils ne se retirent définitivement dans leurs demeures, loin des falaises qui bordent les mers, du cœur sombre des forêts ou des sommets de glace perdus dans les hauteurs du ciel. Nous les représentons généralement sous une apparence humaine, même s'il est vain à présent de chercher à savoir si c'était le cas. Il arriva cependant que les \Ea, ou leurs serviteurs, figèrent leur apparence en accord avec les représentations même qu'en avait faites les Hommes, car elles leur convenaient et elles les aidaient parfois à se connaitre eux-mêmes.

Mais il faut bien comprendre que l’apparence, et plus particulièrement celle d'un \Ea, n’existe pas en soi, elle est un lien éphémère entre celui qui se montre et celui qui regarde. Les Hommes, à leur manière, jouent également de leur apparence, ils ne sont plus capables, comme les \Ea, de changer littéralement de forme, de déformer à ce point leurs traits et les limites de leurs corps, mais ces images, ces facettes qu’ils donnent a voir d’eux-mêmes, cette polysémie d’apparence et de paraitre s’est retranchée en leur cœur, en ce qu’ils racontent ou ce qu'ils façonnent.   

Et dans ce jeu des représentations les dieux avaient souvent perdu les Hommes, qui croyaient alors avoir à faire à plusieurs divinités là où une seule, sous différentes formes, se manifestait sous leurs yeux. Mais peut-on réduire un \Ea, un être d'une envergure sans pareille, doté d'une force archaïque à l'heure où le monde n'était rien, à une seule de ses formes ? Plusieurs dieux n'ont souvent été qu'un, car il ont eu tout à découvrir, et ils se sont faits comme ils ont fait le monde, mais les Hommes ne pouvaient le savoir. Et ils ne cherchaient pas à le savoir, car ils y mettaient du sens et leur attachaient des symboles, et cela était le plus important.

Les \Ea ne sont ni des hommes ni des femmes, ni masculin ni féminins, ils portent en eux tout cela à la fois. Ils se parent du sexe qui leur convient, et en changent régulièrement. Quant à leurs attributs ou leurs représentations chez les Hommes, ils varient d'une culture, d'une époque à l'autre. Ainsi \Oros est souvent représenté comme un homme de grande stature, et on le nomme aussi Aros, mais dans certains contes il apparait sous les traits d'une femme et Aroë est son nom. \Nio est souvent représentée comme une femme, d'une beauté sans pareille, notamment dans l'histoire qui raconte l'apparition de tout ce qui croît en terre sur \Orba. Mais à d'autres reprises elle apparait sous la forme d'un homme et on le nomme Meerildar. \Fercor est le seul \Ea a avoir été toujours représenté, sous forme humaine, sous les traits d'un Homme, violent et irascible, roi des profondeurs de la terre. Ceci est sûrement du au conte qui relate comment il sculpta dans le feu une créature divine dont il tombât éperdument amoureux et qu'il épousa, pour son malheur. D'autres fois les \Ea sont représentés sous les traits d'animaux, ou des formes plus primitives, et nous y reviendront lorsque nous relaterons les récits dans lesquels ils jouèrent un grand rôle.

\begin{paperbox}{De la représentation des \Ea}[lightgray]
Il revient au lecteur de préférer l'une ou l'autre de ces représentations des \Ea, car plus personne aujourd'hui ne pourrait prétendre qu'il connait leur identité fondamentale, et tout ce que l'on raconte sur eux nous viennent d'histoires fragmentées et pour la plupart perdues. 
\end{paperbox}

    

\section{\Orba}

\Cind avait également ordonné à \Carac d'amener non loin de \Naos un étrange objet: il était légèrement oblongue, comme un œuf, et sa surface était noire car elle dévorait tout ce qui venait à elle, même les splendides lumières dorées de \Naos. Cet objet semblait fait de silence et de mort. Une partie de sa surface était brisée et laissait entrevoir ses profondeurs plus sombres encore. Sa peau était faite de nombreuses et épaisses couches dures de chaos cristallisées. %Quelques abimes ouvraient des chemins vertigineux vers son cœur comme de lourdes et profondes blessures, où aux assises de puissants massifs noirs, un feu à l'agonie opposait sans espoir un blanc pur. 
Quelques abimes ouvraient des chemins vertigineux vers son cœur comme de lourdes et profondes blessures où, aux puissantes et noires assises du monde, un feu à l'agonie opposait sans espoir la clarté d'un blanc pur. C'était là le dernier œuf de \Tot, \Carac n'avait pu en raviver la flamme, elle était trop faible pour fleurir et rejoindre le reste des étoiles. Sa lumière blanche était délicate, et si \Carac avait voulu en étirer la moindre couleur elle aurait aussitôt disparu. Elle était la graine noire du monde, et elle était destinée à disparaître.

Les \Ea l'observaient de loin, sa forme obscure se détachait dans le halo doré de \Naos et jamais d'ombre ne parut aussi insensible à sa lumière, il était le héraut funeste d'une armée de ténèbres qui jamais ne viendrait. Pour ceux qui venaient, une telle chose semblait déjà perdue. Ils préféraient alors contempler les étoiles, qui brillaient par milliers autour d'eux sur des distances interminables. \Nio désirait aller à leur rencontre, écouter leurs murmures et goûter à leurs couleurs. \Fercor voulait retourner au cœur de \Naos, parmi les bouillonnements incessants et la hargne  du feu. \Oros ne pouvait s'empêcher de considérer cette masse sombre qui flottait là comme une inexactitude, et elle était laide en son cœur, mais il ne pouvait nier qu'une certaine pitié le saisissait, de ce feu qui était tombé si bas en disgrâce.


\Cind demande aux \Ea de transformer cet objet d'obscurité en un lieu de beautés, car l'espace est achevé par les étoiles. \Oros accepte d'emblée car cette lumière malade l'obsède. \Nio s'y rend à contrecœur. \Fercor se loge au centre de l'objet, au plus près de la flamme. \Oros voit sa tâche comme un devoir, \Nio comme une soumission, \Fercor comme un arrachement, un inconfort qui suscite la colère.\Cind demande à \Carac de surveiller \Nio, et de faire en sorte qu'elle n'aille pas à la rencontre des étoiles. \Ogo suivit ces évènements qui allaient bouleverser l'histoire du monde.

%\Cind croit bien faire se dit-il, il croit que l'univers de \Mey doit être comme une nature morte, un autel de splendeurs en sa mémoire. Mais \Ogo rejetait cette volonté, le monde ne devait pas être un jardin d’éternité, de petites bougies allumées dans la nuit ne servant qu’au souvenir ou à la mémoire, mais un lieu d’affrontement, de construction de la mémoire.

        




%Shuru avait fait de son fragment une etoile filante qu’il enrobait d’un nuage bleuté de faible lumiere et qui parcourait le ciel. Il avait bati son royaume des reves, où tout être imaginant devait aller une fois en rêve.