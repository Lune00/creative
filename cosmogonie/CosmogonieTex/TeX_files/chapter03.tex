\chapter{D'\Orba, la Terre}

\section{Les \Ea}

\Carac, sur ordre de \Cind, déposa le fragment de \Drisst en sa possession au cœur de la grande étoile \Naos. Sa lumière était belle et d'or. Son éclat ne repoussait pas l'obscurité comme les autres étoiles savaient le faire, les obscurités semblaient plutôt lui avoir cédé la place sans résister, par déférence envers une splendeur qu'elles ne pourraient jamais ravir. 

\Naos couva le fragment de \Drisst, et lentement celui se consuma. L'étoile palpita, de vives lumières s'élancèrent dans le monde et les \Ea, à travers la surface dorée de \Naos vinrent à leurs suites: \Oros d'abord, puis \Fercor, et enfin \Nio. En chacun d'eux brulait une part de \Mey et une part d'inconnu.

\section{\Orba}

\Cind avait également ordonné à \Carac d'amener non loin de \Naos un étrange objet: il était légèrement oblongue, comme un œuf, et sa surface était noire car elle dévorait tout ce qui venait à elle, même les splendides lumières dorées de \Naos. Cet objet semblait fait de silence et de mort. Une partie de sa surface était brisée et laissait entrevoir ses profondeurs plus sombres encore. Sa peau était faite de nombreuses et épaisses couches dures de chaos cristallisées. %Quelques abimes ouvraient des chemins vertigineux vers son cœur comme de lourdes et profondes blessures, où aux assises de puissants massifs noirs, un feu à l'agonie opposait sans espoir un blanc pur. 
Quelques abimes ouvraient des chemins vertigineux vers son cœur comme de lourdes et profondes blessures où, aux puissantes et noires assises du monde, un feu à l'agonie opposait sans espoir la clarté d'un blanc pur. C'était là le dernier œuf de \Tot, \Carac n'avait pu en raviver la flamme, elle était trop faible pour fleurir et rejoindre le reste des étoiles. Sa lumière blanche était délicate, et si \Carac avait voulu en étirer la moindre couleur elle aurait aussitôt disparu. Elle était la graine noire du monde, et elle était destinée à disparaître.

Les \Ea l'observaient de loin, sa forme obscure se détachait dans le halo doré de \Naos et jamais d'ombre ne parut aussi insensible à sa lumière, il était le héraut funeste d'une armée de ténèbres qui jamais ne viendrait. Pour ceux qui venaient, une telle chose semblait déjà perdue. Ils préféraient alors contempler les étoiles, qui brillaient par milliers autour d'eux sur des distances interminables. \Nio désirait aller à leur rencontre, écouter leurs murmures et gouter leurs couleurs. \Fercor voulait retourner au cœur de \Naos, parmi les bouillonnements incessants et la hargne  du feu. \Oros ne pouvait s'empêcher de considérer cette masse sombre qui flottait là comme une inexactitude, et elle était laide en son cœur, mais il ne pouvait nier qu'une certaine pitié le saisissait, de ce feu qui était tombé si bas en disgrâce.         





 %Cindara demande a Caracor, a la requete des 3, les Víi (les etres) d’amener l’œuf mort auprès de Naos. Il a une forme légèrement oblongue, comme un œuf, sa surface est noire et lisse, et absorbe la lumiere. Une partie de sa surface est brisée et laisse entrevoir ses profondeurs. Ses couches sont épaisses, faite de multiples peaux de chaos cristalisées, et l’on peut apercevoir par quelques cheminées en son cœur une faible flamme blanche et éclatante bruler. Le feu est trop faible pour étirer et manipuler ses couleurs, ce qui l’éteindrait pour toujours. C’etait là le dernier œuf de Tot, le dernier objet hormi les étoiles. Cindara demanda aux Vii de réaliser là une œuvre aussi belle que celle de Caracor. Cindara voulait en faire une nature morte, comme un autel à la mémoire de Meydra. Mais Ogo rejettait cette volonté, le monde ne devait pas etre un jardin d’éternité, de petites bougies allumées dans la nuit ne servant qu’au souvenir ou à la mémoire, mais un lieu d’affrontement, de construction de la mémoire. Shuru avait fait de son fragment une etoile filante qu’il enrobait d’un nuage bleuté de faible lumiere et qui parcourait le ciel. Il avait bati son royaume des reves, où tout être imaginant devait aller une fois en rêve.