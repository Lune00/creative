\chapter{D'\Orba, la Terre}

\section{Les \Ea}

\Carac, sur ordre de \Cind, plaça délicatement le fragment de \Drisst qu'il avait en sa possession au cœur de la grande étoile \Naos. Sa lumière était belle et d'or. Son éclat ne repoussait pas l'obscurité comme les autres étoiles savaient bien le faire, les obscurités semblaient plutôt lui avoir céder la place et elles s'étaient même inclinées en se retirant, par politesse pour une splendeur qu'elles ne pourraient jamais ravir. 

\Naos couva le fragment de \Drisst, et lentement celui se consuma. L'étoile palpita, de vives lumières s'élancèrent dans le monde et les \Ea, à travers la surface dorée de \Naos vinrent à leurs suites: \Oros d'abord, puis \Fercor, et enfin \Nio. En chacun d'eux brulait une part de \Mey et une part d'inconnu.

\section{\Orba}

\Cind avait également ordonné à \Carac de placer devant eux, non loin de \Naos un étrange objet: il était légèrement oblongue, comme un œuf, et sa surface était noire car elle dévorait tout ce qui venait à elle, même les splendides lumières dorées de \Naos. Cet objet semblait fait de silence et de mort. Une partie de sa surface était brisée et laissait entrevoir ses profondeurs plus sombres encore. Sa peau était faite de nombreuses et épaisses couches dures de chaos cristallisées. Quelques abimes ouvraient des chemins vertigineux vers son cœur comme de lourdes et profondes blessures, où aux assises de puissants massifs noirs, un feu à l'agonie opposait sans espoir un blanc pur.         





 %Cindara demande a Caracor, a la requete des 3, les Víi (les etres) d’amener l’œuf mort auprès de Naos. Il a une forme légèrement oblongue, comme un œuf, sa surface est noire et lisse, et absorbe la lumiere. Une partie de sa surface est brisée et laisse entrevoir ses profondeurs. Ses couches sont épaisses, faite de multiples peaux de chaos cristalisées, et l’on peut apercevoir par quelques cheminées en son cœur une faible flamme blanche et éclatante bruler. Le feu est trop faible pour étirer et manipuler ses couleurs, ce qui l’éteindrait pour toujours. C’etait là le dernier œuf de Tot, le dernier objet hormi les étoiles. Cindara demanda aux Vii de réaliser là une œuvre aussi belle que celle de Caracor. Cindara voulait en faire une nature morte, comme un autel à la mémoire de Meydra. Mais Ogo rejettait cette volonté, le monde ne devait pas etre un jardin d’éternité, de petites bougies allumées dans la nuit ne servant qu’au souvenir ou à la mémoire, mais un lieu d’affrontement, de construction de la mémoire. Shuru avait fait de son fragment une etoile filante qu’il enrobait d’un nuage bleuté de faible lumiere et qui parcourait le ciel. Il avait bati son royaume des reves, où tout être imaginant devait aller une fois en rêve.