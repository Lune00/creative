\chapter{L'âge des étoiles}

\subsection{\Cind}

\Cind se retrouva seul et dans un état d'affliction qu'il n'avait jamais connu. \Mey lui avait fait par de sa volonté de bâtir une idée qui se développerait d'elle même, sous sa propre musique, où eux, l'être Unique, n'aurait que à coeur de la laisser évoluer. \Cind avait fait le serment à \Mey de ne jamais détruire quoi que ce soit dans l'univers. \Cind appela cet univers \dreyma et exauça les derniers souhaits de sa part disparue. \Cind n'avait pas la créativité de \Mey il s'imprégna alors de sa propre philosophie et décida de faire ré apparaitre la lumière. Il créa \carac, un être divin sans paroles et au but unique: faire reculer les ténèbres et ramener la lumière. \carac apparut dans \dreyma sous la forme d'un spectre, \Cind lui donna un immense marteau d'un noir sans reflet car il remarqua que seuls les oeufs de \Tot avaient survécu au souffle et qu'en eux couvaient la lumière du chaos originel. \Boromu, dans sa grande sagesse, avaient protégé les oeufs de ses puissantes flammes bleues qui absorbèrent le souffle de \Cind.  \carac se mit à la tâche, il alla d'œuf en œuf pour en fracasser la coquille et libérer l'éclat du feu à présent inerte et la lumière se remis à arpenter les espaces vierges du vide. De son souffle il sculpta leur éclat, et de ses mains en ajusta les flammes, en étirant les couleurs comme des fils invisibles jusqu'à ce que la composition soit harmonieuse. Il fit éclore les étoiles comme les fleurs d'un jardin de pénombres. Il les déplaça et les arrangea pour éclairer l'univers de la façon la plus harmonieuse. Les couleurs des flammes, bleues et rouges, vertes et jaunes s'enroulaient les unes dans les autres et de nouvelles couleurs naquirent. \carac, en hommage à \Boromu, rassembla les flammes de dix œufs en une, à laquelle il donna un bleu puissant, lumineux et calme, et forma l'étoile la plus grande que le monde ait connue. Il l'alluma d'un coup de marteau bien ajusté, elle brilla jusqu'à la fin des temps. 

En silence, infatigable, son lourd marteau noir fracassait de manière régulière les œufs aux quatre coins du monde. Le rythme régulier de ses coups faisait vibrer la structure de l'espace, et les étoiles pulsaient calmement à leur rythme. Du marteau de \carac naquit le temps, première horloge connue du monde, dont les échos résonnaient dans l'immensité du vide. La musique de son marteau, à la portée infinie, était la seule voix connue de \carac, la première voix du temps. Le météore issu de \Mey, Dristt, dansait, libre, parmi les étoiles qui s'allumaient peu à peu. Elle décrivait des courbes gracieuses et voguait avec grâce et légèreté dans l'espace. En passant dans leur voisinage sa trajectoire s'illuminait d'une majestueuse trainée scintillante, reflétant des couleurs nouvelles et resplendissantes.


\subsection{\Ogo}

Les œufs de \Tot survécurent à l'\Extinction et parmi l'un d'eux \Ogo avait réussi à se cacher. Il était cependant prisonnier de la coquille épaisse et affaibli ayant perdu son mouvement. Seule sa volonté était intacte et il chercha par toutes les ruses à corrompre \carac pour qu'il vienne le délivrer. Mais \carac était une ombre noire, muette et sourde à tout appel, à toute injonction. Il était absorbé par sa tâche où seul son art d'allumer les étoiles lui donnait la tendresse d'un être avec lequel on pouvait cultiver des liens. Mais \carac n'avait des liens qu'avec les étoiles. \carac finit par se présenter devant l'oeuf où \Ogo se cachait. Mais \carac s'en détourna car il sentit que la lumière et la chaleur de cette flamme corrompraient son jardin de lumière. \Ogo avait eu beau dissimuler du mieux qu'il put sa volonté et son chaos indomptable il n'avait pas réussi à berner \carac qui continua son chemin et reprit son oeuvre inlassable de jardinier des étoiles. 

Alors \Ogo eut une idée car jamais le désespoir ne pouvait l'atteindre. \auga, autrefois tempête de couleurs et de flots, à la beauté mortelle était figée dans une déformation de l'espace lugubre et sans beauté. L'espace s'était figé au point culminant de sa douleur, dans une grimace d'effroi, recroquevillé sur lui même en crevasses et en plis qui se recouvraient dans un dédale d'étrangeté. \auga avait perdu ses couleurs et le dégout qu'elle inspirait était à la hauteur de l'émerveillement qu'elle suscitait. Son pouvoir hypnotique était passé des mains de la splendeur à celle de l'horreur indicible. Autour de son noyau l'espace s'était creusé en vallées qu'aucune lumière ne viendrait jamais éclairer. Les replis infinis de sa courbure exacerbée abritaient les ténèbres. \auga avait été si violente qu'elle avait déchiré la structure de l'espace et l'univers s'était ouvert en un endroit sur des espaces qu'aucune imagination ne pouvait arpenter. \Ogo ressentit cette fissure et fit rouler inlassablement ses appels dans les méandres d'\auga jusqu'à ce qu'ils tombent par la fissure vers ces espaces étranges et insondables. Et l'un de ses appels fut entendu. Une créature surgit des dimensions inconnues et se glissa à travers la fissure du monde. \Ogo la nomma \Shuru.

--- \auga m'apporte enfin, en ta personne, ce que j'ai tant désirée. Peu importe qui tu es et quels sont tes desseins. Redonne-moi la liberté que j'ai conquise. Il n'y a rien de pire que de briller, isolé, dans le néant. \n

--- Quelle liberté t'as donc apportée \auga? Celle d'être maitre de tes désirs et d'arpenter les parois de ta cage alors que l'espace est si vaste, enfant de \Mey?\n

--- Je ne suis l'enfant de personne. Ma volonté est là d'où je viens, et ce que je suis je ne le dois à personne. Je préfère ma captivité à ma \textit{liberté} d'alors, où j'arpentais une cage sans bords et sans fin. A présent ma prison a des murs qu'il est possible de briser. Que faire d'une volonté dont on ne peut rien faire? A quoi bon guider les rivières de flammes si la seule chose qui nous distingue d'elles est d'avoir conscience que nous nous écoulons pour que rien, jamais, ne se produise ?\n

--- Tu aurais pu remplir ton rôle et attendre ton moment. Tu ne sauras jamais ce que \Mey aurait pu te donner. \n

--- Je te le répète, je ne suis l'enfant de personne. \Mey ne m'a pas donné vie et je ne luis dois rien, il n'a fait que révéler ce que j'ai toujours été. En me condamnant à l'inaction \Mey s'est opposé à ma volonté. Être est une chimère. Finir par être, c'est paraître. Je ne \textit{suis} rien, j'agis et ce que je suis n'est que le reflet de mes actes. Le reflet seulement. \Mey a fait l'erreur de croire que je lui appartenais. Personne, aucun dieu, aucune puissance, ne pourra me posséder ou m'attribuer un rôle. Mon rôle je l'ai toujours choisi, mon destin n'est écrit nulle part et n'est nulle part à écrire. De moi il n'y a que l'ardeur des flammes. Agir et surgir. Rugir. Je suis de feu et le feu n'appartient pas à celui qui a fait jaillir la première étincelle. Libère moi à présent.

--- \Mey m'a aussi donné vie, mais il n'a jamais été capable d'achever l'idée qu'il avait de moi. Il m'a laissé parcourir ses pensées, voyager sur toutes les voies de son imagination, il voulait que je perce le secret des visions auxquelles il était incapable de donner forme. Voilà des éternités que j'arpente les paysages de sa pensée, et puisqu'il m'a fait ainsi la curiosité des imaginaires est devenue ce que je suis, ma propre quête. A présent que \Mey a cessé d'exister mon existence est sans but. Pourquoi ne mettrai-je pas un terme à ton existence en te laissant croupir dans ton œuf jusqu'à la fin des éternités, toi qui est si sûr de toi et de ta volonté et qui pourtant est aujourd'hui sans pouvoir?

--- \Mey ne m'a pas donné vie, il n'a fait que la découvrir. Là où tu te trompes c'est que mon pouvoir n'est pas éteint et que mon feu couve. Tu me libéreras car tu ne pourras résister à la tentation de satisfaire la curiosité qui te dévore. Si \Mey t'offrait des espaces imaginaires insondables ici je t'offrirai des espaces où des forces, animées par leurs feus intérieurs, s'entrechoqueront comme jamais tu n'as pu voir.

--- La prétention qui t'anime a le double effet de m'exaspérer et de m'exalter. Si tu crois un jour pouvoir rivaliser avec le pouvoir de création de \Mey tu te trompes. Les idées sont aussi réelles que le feu que tu te plais tant à couver.

--- Seulement les idées n'ont pas le pouvoir d'explorer tous les possibles. Tu serais surpris de voir ce qu'il peut arriver lorsque de simples flammes ont une volonté. Restreindre les possibles en créent de nouveaux que nul n'aurait été capable d'imaginer. Et aucun dieu ne peut accorder de l'importance à tout ce que son esprit est capable de produire.

--- Si tu crois que \Mey s'est seulement contenté de penser avant \textit{ta venue} tu te trompes, répondit \Shuru. Cet univers n'est issu que d'un songe, d'une rêverie, des petites éternités à peine ont passé depuis sa formation. J'ai parcouru toutes les œuvres de l'Unique, celles mises à l'épreuve comme celles, trop étranges et sans formes, restées en suspens. J'ai parcouru des mondes en train de naitre, d'autres prompt à disparaitre. J'ai vu des choses que ton esprit arrogant et étroit serait bien incapable de concevoir.

--- Contrairement à \Mey ce qui a été n'a aucune importance pour moi, et rien n'est reproduit, tout est produit à nouveau. Si mon esprit est plus étroit alors ma volonté, elle, est plus grande. D'ailleurs, \Mey n'a rien vu de plus beau qu'\auga. Et cet univers sera son dernier. Plus jamais il n'y en aura. \Cind est incapable d'en penser un autre. Souhaites-tu voir la dernière œuvre de \Mey finir en un long soupir administré par un dieu à l'agonie?

--- Je t'accorde que cet univers portait une place particulière dans l'esprit de \Mey, comme je n'en avais jamais vu auparavant. Je m'interroge sur l'importance qu'il lui accordait, étant donné qu'à présent il me semble qu'il sera impossible de le découvrir. Il n'est plus qu'une idée avortée, laissée en proie aux ombres et à la lumière. \Mey était puissant mais si fragile. Lorsque je parcourais ses pensées je ne faisais que les effleurer tant j'avais peur qu'elles se brisent à mon approche.

  
  
  
  
%"Il m'est impossible de te libérer, seul le feu de ce monde peut briser ton entrave". Qu'aurais-je a y gagner? Je vois qu'en toi couve la puissance de \Mey. Moi même je suis l'un de ses enfants, il n'a jamais été capable d'achever l'idée qu'il avait de moi. Il m'a doté de nombreux pouvoirs notamment de parcourir ses pensées, il voulait que je fabrique ce qu'il n'avait jamais pu penser. Tu l'as détruit aujourd'hui je erre sans but. Pourquoi ne mettrai-je pas un terme à ton existence en te laissant croupir dans ton oeuf?" \Ogo lui parla du météore.   