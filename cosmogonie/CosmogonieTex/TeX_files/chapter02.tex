\chapter{L'âge des étoiles}

\subsection{\Cind}

\Cind se retrouva seul et dans un état d'affliction qu'il n'avait jamais connu. \Mey lui avait fait par de sa volonté de bâtir une idée qui se développerait d'elle même, sous sa propre musique, où eux, l'être Unique, n'aurait que à coeur de la laisser évoluer. \Cind avait fait le serment à \Mey de ne jamais détruire quoi que ce soit dans l'univers. \Cind appela cet univers \Dreyma et exauça les derniers souhaits de sa part disparue. \Cind n'avait pas la créativité de \Mey il s'imprégna alors de sa propre philosophie et décida de faire ré apparaitre la lumière. Il créa \Carac, un être divin sans paroles et au but unique: faire reculer les ténèbres et ramener la lumière. \Carac apparut dans \Dreyma sous la forme d'un spectre, \Cind lui donna un immense marteau d'un noir sans reflet car il remarqua que seuls les oeufs de \Tot avaient survécu au souffle et qu'en eux couvaient la lumière du chaos originel. \Boromu, dans sa grande sagesse, avaient protégé les oeufs de ses puissantes flammes bleues qui absorbèrent le souffle de \Cind.  \Carac se mit à la tâche, il alla d'œuf en œuf pour en fracasser la coquille et libérer l'éclat du feu à présent inerte et la lumière se remis à arpenter les espaces vierges du vide. De son souffle il sculpta leur éclat, et de ses mains en ajusta les flammes, en étirant les couleurs comme des fils invisibles jusqu'à ce que la composition soit harmonieuse. Il fit éclore les étoiles comme les fleurs d'un jardin de pénombres. Il les déplaça et les arrangea pour éclairer l'univers de la façon la plus harmonieuse. Les couleurs des flammes, bleues et rouges, vertes et jaunes s'enroulaient les unes dans les autres et de nouvelles couleurs naquirent. \Carac, en hommage à \Boromu, rassembla les flammes de dix œufs en une, à laquelle il donna un bleu puissant, lumineux et calme, et forma l'étoile la plus grande que le monde ait connue, \Boromil. Il l'alluma d'un coup de marteau bien ajusté, elle brilla jusqu'à la fin des temps. 

En silence, infatigable, son lourd marteau noir fracassait de manière régulière les œufs aux quatre coins du monde. Le rythme régulier de ses coups faisait vibrer la structure de l'espace, et les étoiles pulsaient calmement à leur rythme. Du marteau de \Carac naquit le temps, première horloge connue du monde, dont les échos résonnaient dans l'immensité du vide. La musique de son marteau, à la portée infinie, était la seule voix connue de \Carac, la première voix du temps. Le météore issu de \Mey, \Drisst, dansait, libre, parmi les étoiles qui s'allumaient peu à peu. Elle décrivait des courbes gracieuses et voguait avec grâce et légèreté dans l'espace. En passant dans leur voisinage sa trajectoire s'illuminait d'une majestueuse trainée scintillante, reflétant des couleurs nouvelles et resplendissantes.


\subsection{\Ogo, Shuru et le réel}

Les œufs de \Tot survécurent à l'\Extinction et parmi l'un d'eux \Ogo s'était caché. Il était cependant prisonnier de son épaisse écorce et affaibli. Seule sa volonté était intacte et il chercha par toutes les ruses à corrompre \Carac pour qu'il vienne le délivrer. Mais \Carac était une ombre noire, muette et sourde à tout appel. Il était absorbé par sa tâche où seul son art d'allumer les étoiles lui donnait la tendresse d'un être avec lequel on pouvait cultiver des liens. Mais \Carac n'avait de liens qu'avec les étoiles. Lorsque \Carac se présenta devant l'œuf renfermant \Ogo, il s'en détourna, car il sentit que la lumière et la chaleur de cette flamme corrompraient son jardin de lumière. \Cind l'avait mis en garde. \Ogo avait eu beau dissimuler du mieux qu'il put sa volonté et son chaos indomptable, il n'avait pas réussi à tromper \Carac qui continua son chemin et reprit son œuvre, inlassable, de jardinier des étoiles. 

%son art d'allumer les étoiles lui donnait l'apparence d'un etre tendre avec lequel on pouvait cultiver des liens

Alors \Ogo eut une idée car jamais le désespoir ne pouvait l'atteindre. \Auga, autrefois tempête de couleurs et de flots, à la beauté mortelle était figée dans une déformation de l'espace lugubre et sans beauté. L'espace s'était figé au point culminant de sa douleur, dans une grimace d'effroi, recroquevillé sur lui même en crevasses et en plis qui se recouvraient dans un dédale d'étrangeté. \Auga avait perdu ses couleurs et le dégout qu'elle inspirait était à la hauteur de l'émerveillement qu'elle suscitait. Son pouvoir hypnotique était passé des mains de la splendeur à celle de l'horreur indicible. Autour de son noyau l'espace s'était creusé en vallées qu'aucune lumière ne viendrait jamais éclairer. Les replis infinis de sa courbure exacerbée abritaient les ténèbres. \Auga avait été si violente qu'elle avait déchiré la structure de l'espace et l'univers s'était ouvert en un endroit sur des espaces qu'aucune imagination ne pouvait arpenter. \Ogo ressentit cette fissure et fit rouler inlassablement ses appels dans les méandres d'\Auga jusqu'à ce qu'ils tombent par la fissure vers ces espaces étranges et insondables. Et l'un de ses appels fut entendu. Une créature surgit des dimensions inconnues et se glissa à travers la fissure du monde. \Ogo finit par apprendre qu'elle se nommait \Shuru.

%%A REE CRIRE TODO!!!!!
% Shuru dira a Ogo qu'il lui a donné les mots, le langage.
--- \Auga m'apporte à nouveau ce que j'ai tant désirée. Redonne-moi la liberté, celle, véritable, que j'ai conquise. Il n'y a rien de pire que de briller, isolé, dans le néant. \n

--- Quelle liberté t'as donc apportée \Auga? Elle ne fait qu'écrouler les portes et dresser des murs enfant de \Mey. \n

--- Je ne suis l'enfant de personne. Ma volonté est là d'où je viens, et ce que je suis je ne le dois à personne. Je préfère ma captivité à ma \textit{liberté} d'alors, où j'arpentais une cage sans bords et sans fin. A présent ma prison a des murs et les murs n'existent que pour être brisés. Que faire d'une volonté dont on ne peut rien faire? A quoi bon exister si la seule chose qui nous distingue des rivières du chaos est d'avoir conscience que nous nous écoulons pour que rien, jamais, ne se produise ?\n

--- Tu n'as aucune idée de ce que signifie jamais. Tu aurais pu remplir ton rôle et attendre ton moment. \Mey t'estimais et tu étais pour lui ce qu'il y avait de plus précieux à apprivoiser. D'être merveilleux en devenir te voilà réduit à l'état d'une flamme qui ne brille pour rien.\n

---  Être pour \Mey? Mais je ne \textit{suis} rien, j'agis et ce que je suis n'est que le reflet de mes actes. Le reflet seulement. Être est une chimère. Finir par être, c'est paraître. Il a fait l'erreur de croire que je lui appartenais alors que de moi il ne possédait qu'une image. \Mey ne m'a pas donné vie, il n'a fait que la révéler. Personne, aucun dieu, aucune puissance, appelle cela comme bon te semble, ne pourra me posséder ou m'attribuer un rôle. Mon rôle je le choisis toujours, mon destin n'est écrit nulle part et n'est nulle part à écrire. J'ai surgi des flammes et par elles j'ai agi. Agir et surgir. Rugir. Je suis de feu, ça je ne l'ai pas choisi, mais il m'appartient désormais de choisir ce qui est en proie aux flammes.

% En me condamnant à l'inaction \Mey s'est opposé à ma volonté.   

--- \Mey m'a aussi donné vie. Il m'a laissé parcourir ses pensées, voyager sur toutes les voies de son imagination, il voulait que je perce le secret des visions. Voilà des éternités que j'arpente les paysages de sa pensée, et puisqu'il m'a fait ainsi la curiosité des imaginaires, m'exposer aux images est devenue ce que je suis, ma propre quête. A présent que \Mey a cessé d'exister mon existence est sans but. Pourquoi ne mettrai-je pas un terme à la tienne en te laissant croupir dans ton œuf jusqu'à la fin des éternités ?

--- Je te le répète, \Mey ne m'a pas donné vie, il n'a fait que la découvrir. Tu te trompes, mon pouvoir n'est pas éteint et mon feu couve. Tu me libéreras car tu ne pourras résister à la tentation de satisfaire la curiosité qui te dévore. Si \Mey t'offrait des espaces imaginaires insondables ici je t'offrirai des espaces où des forces, animées par leurs désirs, leurs propres volontés, s'entrechoqueront comme jamais tu n'as pu en voir.

--- La prétention qui t'habite a le double effet de m'exaspérer et de m'exalter. Si tu crois un jour rivaliser avec le pouvoir de \Mey tu te trompes. Cet univers n'est même pas issu d'un songe, mais d'une rêverie, des petites éternités à peine se sont écoulées depuis sa formation. J'ai parcouru toutes les œuvres de l'Unique, celles mises à l'épreuve comme celles, trop étranges et sans formes, restées en suspens. J'ai parcouru des mondes en train de naitre, d'autres prompt à disparaitre. J'ai été aux prises avec des choses que ton esprit arrogant et étroit serait bien incapable de concevoir. A présent tout cela est perdu.

--- Seulement le songe ne permet pas explorer tous les possibles. Restreindre les possibles en créent de nouveaux que nul être imaginant ne puisse entrevoir. Et aucun dieu, fût-ce \Mey, ne peut accorder de l'importance à tout ce que son esprit est capable de produire.

--- Tu prétends qu'il peut se produire ici davantage que dans l'infinité des songes ? 

--- Je ne le prétends pas, je l'affirme. Contrairement à toi, ce qui a été n'a aucune importance pour moi, et rien n'est à reproduire, tout est à produire à nouveau. Si mon esprit est plus étroit alors ma volonté, elle, est plus grande. D'ailleurs, \Mey n'a rien vu de plus beau qu'\Auga. Et cet univers sera son dernier. Plus jamais il n'y en aura. \Cind est incapable d'en penser un autre. Souhaites-tu voir la dernière œuvre de \Mey finir en un long soupir administré par un dieu à l'agonie?

--- Je t'accorde que cet univers avait une place particulière dans l'esprit de \Mey. Je m'interroge sur l'importance qu'il lui accordait. A présent, il n'est plus qu'une pensée morte, qui a fini par tomber dans ce que je nomme le \reel, l'espace abandonné de l'imaginaire, l'imagination figée.  Ce que j'ai toujours redouté c'est précisément ceci, le réel. Le réel m'inquiète, c'est le lieu que l'imaginaire a déserté, qui peu à peu prend son autonomie, et où la causalité étend comme une maladie son emprise, et vient avec sa monotonie mortifère administrer les modalités d'interaction entre les produits imaginaires. Cette causalité orchestre le ballet des images mortes. \Mey était et restera le seul qui pouvait créer à partir de rien. C'est pour cette raison d'ailleurs qu'il me créea. Il voulait que je trouve l'origine de ces images qui lui venaient sans cesse, car il ne croyait pas en ce don, pour lui cette capacité devait prendre racine dans quelque chose qui lui pré-existait. Étant lui même part de l'être Unique, comme lui rappelait Cindara, il ne pouvait trouver de réponse à cette angoisse, hormis dans les images meme qu'il produisait. Cette angoisse finit par le dévorer, l'affaiblir.   
%Le lien de causalité que je recherchais est à présent brisé. Ce n'est pas celui qui unit les images entre elles, mais celui entre les images et ce qui leur a pré existé. 
Et cette réponse ne pouvait résider que dans l'imaginaire vibrant de \Mey. Mais comment trouver les indices d'une quelconque origine lorsque l'on arpente une infinité d'images explorant tous les chemins de l'imaginaire? En vérité, j'ai vite compris qu'il m'était impossible de répondre à cette question, du moins de cette manière. Alors j'ai fini par arpenter ses images pour mon propre plaisir, feignant de trouver des commencements de réponse. Je ne sais pas s'il me crut ne serait-ce qu'une fois. Mais cela le rassurait. Sinon j'aurais sombré dans la plus obscure des folies.  \Auga n'était rien comparée à ce que j'ai pu éprouver au cours de mes voyages. Ce monde a si peu de dimensions... Et à présent tu m'as condamné pour toujours à ne vivre que d'un fragment dérisoire de son imaginaire figé.

%La causalité qu'interesse Shuru n'est pas celle entre les images elle même mais entre les images et leur source.
%Shuru cherche un lieu de causalité (quelle est l'origine des visions de Meydra? Et il hait la causalité) Comment trouver le lien de causalité entre des images qui se succedent indéfiniment?

--- Donne du temps au réel et il te surprendra davantage que n'importe quel imaginaire. Le réel est le lieu de la liberté, là où le sens et les représentations nous appartiennent.

--- Tu ne saisis pas ce qui a été perdu. 

--- Je ne m'enquis pas de ce qui est perdu mais de ce qui est gagné, mais cessons ces bavardages, libère moi.

%--- Il en est hors de question. Le réel, tout cela, quoique tu fasses tu ne répareras jamais l'irréparable, la réduction de l'imagination à un seul possible, une seule ligne d'évènements.

Pour le convaincre \Ogo du le mettre dans un secret dont il n'aurait voulu rester maître. Il lui apprit que \Mey n'avait pas encore complètement disparu, qu'une part de lui parcourait encore les espaces sombres du réel sous la forme libre d'un météore et qu'il était possible de retrouver ce qu'il désirait tant. Avare d'imagination, \Shuru accepta de libérer \Ogo pour qu'il l'aide à trouver \Drisst. Il était impossible pour \Shuru de briser lui même l'épaisse carapace de l'œuf dans lequel était enfermé \Ogo. Seul \Carac pouvait la briser de son marteau. Comme \Carac était incorruptible, \Shuru s'en prit alors à ce qu'il avait de plus précieux. Invisible, il se mit à détruire les étoiles une par une. Leurs couleurs majestueuses, soigneusement arrangées par \Carac, disparaissaient sans un soupir et laissaient seulement derrière elles le souvenir de la lumière, aussitôt englouti par le flot infatigable des ténèbres. \Carac, l'ombre sans mots, artisan des étoiles, éprouva une grande tristesse. Et Les étoiles continuaient de disparaitre, une par une. \Carac arrivait toujours trop tard, l'étoile à l'agonie vers laquelle il se déplaçait avait déjà disparu. Aussi sa tristesse grandit en même temps qu'une colère sombre. \Carac écouta attentivement la pulsation de ses étoiles, la seule langue qu'il ne pourrait jamais comprendre. Et lorsqu'il entendit à nouveau leurs cœurs s'arrêter il comprit qu'une force était à l'œuvre et il commença à suivre ses déplacements en écoutant les complaintes des étoiles. Il fut difficile à \Carac de laisser mourir ses étoiles pour approcher l'être qu'il s'était juré de détruire. Bientôt plus de la moitié de ses étoiles disparut et le silence tombait peu à peu sur l'espace, \Carac se retrouvait toujours plus seul, et son ombre se fondait peu à peu dans les ténèbres. Enfin, il fut capable, grâce aux vibrations des étoiles survivantes, de sentir où la force allait frapper ensuite. Il saisit son marteau et à l'instant où une autre étoile mourut le lança de toutes ses forces dans cette direction. Son marteau éventra les ténèbres, en projetant dans sa course une lumière si forte qu'elle parcourt encore l'espace aujourd'hui. Il fondit en direction de l'astre assassiné mais il ne rencontra rien à briser là et il continua sa course dans sa fureur. Il finit par rencontrer, dans sa course folle, un corps silencieux et abandonné dont il fit voler en éclat l'écorce: \Ogo était libre, même s'il failli disparaître lui aussi dans l'impact tellement il fut violent. \Ogo et \Shuru s'enfuirent et laissèrent \Carac à son amertume et à son chagrin. \Carac ramassa son marteau et se remit à sa tâche. Il alla réveiller les œufs encore endormis dans les moindres recoins du monde pour redonner vie à la lumière. Cette deuxième génération d'étoiles inonda l'espace de couleurs plus profondes et pures. 




   
   %\Mey était puissant mais si fragile. Lorsque je parcourais ses pensées je ne faisais que les effleurer tant je craignaient qu'elles puissent se briser à mon approche. Je les parcourais en silence de peur d'effrayer les paysages qui prenaient forme devant moi. La plus noire de ses dépressions pouvaient produire des mondes si étranges et obscures.

\subsection{La capture de \Drisst}

\Ogo et \Shuru se mirent en quête de \Drisst. Lorsqu'ils le trouvèrent ils se rendirent compte qu'il était impossible de l'attraper par la force. Sa course était si libre et inspirée, sa vitesse si grande, qu'il était impossible de s'en approcher. Seule sa trainée scintillante à l'approche des étoiles permettait de le localiser facilement. Et à présent que l'espace s'était appauvri en lumière, \Drisst paraissait plus que jamais hors d'atteinte. Même à deux \Ogo et \Shuru devaient se rendre à l'évidence qu'aucune de leur stratégie ne marcherait. \Shuru, furieux, commença à fustiger \Ogo. Mais \Ogo se souciait peu de \Shuru, toute son attention était vouée à \Drisst. Sa convoitise était plus grande que celle de \Shuru mais contrairement à lui il la dominait complètement, capture \Drisst n'était qu'un obstacle à franchir, un passage à emprunter et non un but.

\Ogo se mit alors à suivre \Drisst comme il put. Il cessa de chercher à l'attraper mais à plutôt à l'apprivoiser. \Ogo passa des éternités à apprendre à se déplacer comme lui. Il y mit toute son énergie, toutes ses forces, toute sa volonté. Enfin il fut capable de ressentir où \Drisst irait en suivant son propre désir d'aller. Il ordonna à \Shuru d'aller en un coin reculé du monde, qu'aucune lumière n'avaient encore éclairé, et d'y attendre. \Drisst finit par s'y rendre et \Shuru le recueillit, et parvint à stopper sa course. Le météore palpitait et \Shuru ressentit en lui, ému, l'imaginaire vibrant de \Mey. Mais \Shuru, absorbé par son désir n'avait pas ressenti que lors de son attente, immobile, \Carac avait fini par le retrouver. Au moment où \Shuru récupéra \Drisst \Carac apparut devant lui, comme un immense voile plus sombre encore que l'obscurité, et lui assena un violent coup de marteau. \Shuru failli disparaitre à son tour. Et si le coup ne le tua pas, il fit voler en éclat \Drisst, en trois fragments. \Carac récupéra le plus gros d'entre eux, \Ogo arriva en toute vitesse et s'empara du deuxième fragment. Le plus petit d'entre eux resta en la possession de \Shuru. La douleur seule du coup reçu par \Carac lui appris qu'il serait impossible de récupérer ce fragment manquant.

\Ogo couva de son feu le fragment de \Drisst et éprouva dans son entier le pouvoir d'imagination parcourir son corps. Une fois le fragment entièrement consumé, son premier acte fut de s'incarner sous une forme qui lui convenait mieux, un dragon(?)


\Cind avait suivi tout ce qu'il s'était passé dans \Dreyma. Il savait que \Ogo était toujours là. Il ordonna à \Carac de faire brûler le fragment en sa possession dans la plus belle de ses étoiles. \Shuru se réfugia dans les ténèbres et entreprit la reconstruction des imaginaires de \Mey. 

\Ogo craignait, comme \Shuru, le marteau de \Carac. \Carac pendant ce temps, avait rassemblé les étoiles en bouquets, dans certaines régions pour qu'elles ne soient plus isolées, et l'ardeur de leurs flammes fut renouvelée. Alors qu'il évoluait à travers elles --- on ne saura jamais si c'est pour se repentir et préserver son existence, ou comme premier acte créatif --- inspiré par sa danse avec \Drisst, \Ogo fit don du mouvement à quelques étoiles et leur dessina des trajectoires incurvées sur elles-même, délaissant les lignes droites au profit de courbes fermées. \Carac fut enchantée à la contemplation de cette danse de lumières, et mis chacune de ses étoiles en mouvement avec une palette de variations aussi sensible que la palette de ses couleurs. Des étoiles regroupées en bouquet, à l'éclat vif, murmurants entre elles d'une voix musicale que seul \Carac pouvait entendre, se dégageaient une beauté envoutante et renouvelée. Les premières structures de \Dreyma étaient le premier chef d'œuvre qu'une volonté en quête de sens pouvait contempler. \Ogo se mit à respecter \Carac et il l'idée de devoir craindre son marteau ne le traversa plus jamais. \Boromil, la plus grande des étoiles, la géante bleue, présentait une grande tache pourpre, un immense orage rappelant les éclairs qui, jadis, parcouraient le corps de \Boromu. Cette tache dérivait lentement sur sa surface bleutée. Lorsqu'elle pointait en direction de la dernière étoile éteinte par \Shuru, les étoiles, sous l'ordre de \Carac, changeaient un temps leurs couleurs et rayonnaient d'une couleur pale et sourde, un mélange de bleu et de pourpre. Cette lumière si particulière, écartait l'obscurité tout en disant même la fragilité de la lumière face aux ténèbres devant lesquelles elles seules se dressent et se dresseraient. Les vastes espaces sans frontières du monde semblaient alors se resserrent quelques instants, en un lieu intime, où les chuchotements de voix basses glissent entre les fragiles lueurs de bougies, et cette lumière rappelait pour toujours la tristesse des étoiles, et incidemment, la liberté retrouvée d'\Ogo.

%Donner la date dans le calendrier des hommes et la durée. Cet évenement apparaitra dans le jeu.