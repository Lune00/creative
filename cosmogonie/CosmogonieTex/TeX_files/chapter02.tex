\chapter{L'âge des étoiles}

\subsection{\Cind}

\Cind se retrouva seul et dans un état d'affliction qu'il n'avait jamais connu. \Mey lui avait fait par de sa volonté de bâtir une idée qui se développerait d'elle même, sous sa propre musique, où eux, l'être Unique, n'aurait que à coeur de la laisser évoluer. \Cind avait fait le serment à \Mey de ne jamais détruire quoi que ce soit dans l'univers. \Cind appela cet univers \dreyma et exauça les derniers souhaits de sa part disparue. \Cind n'avait pas la créativité de \Mey il s'imprégna alors de sa propre philosophie et décida de faire ré apparaitre la lumière. Il créa \carac, un être divin sans paroles et au but unique: faire reculer les ténèbres et ramener la lumière. \carac apparut dans \dreyma sous la forme d'un spectre, \Cind lui donna un immense marteau d'un noir sans reflet car il remarqua que seuls les oeufs de \Tot avaient survécu au souffle et qu'en eux couvaient la lumière du chaos originel. \Boromu, dans sa grande sagesse, avaient protégé les oeufs de ses puissantes flammes bleues, il avait absorbé le souffle de \Cind.  \carac se mit à la tâche, il alla d'oeuf en oeuf pour en briser la coquille et libérer l'éclat du feu à présent inerte et la lumière se remis à arpenter les espaces infins. Il sculpta leur éclat, en attisa les flammes et leur redonna leur éclat d'antan en ajustant l'équilibre des couleurs comme les fleurs d'un jardin d'obscurité. Il les déplaça et les arrangea pour éclairer l'univers de la façon la plus harmonieuse selon n'importe quel point de vue. Les couleurs des flammes, bleues et rouges, vertes et blanches, jaunes et cyan s'enrobaient les unes dans les autres et de nouvelles couleurs naquirent. \carac, en hommage à \Boromu, brisa la coquille de dix oeufs pour en former un seul, d'un bleu puissant et calme, et forma l'étoile la plus grande que le monde ait connue. Il l'alluma d'un coup de marteau bien ajusté et elle brille encore. En silence, son lourd marteau enrobé de flammes du chaos originel frappait avec une régularité infatigable les oeufs aux quatres coins du monde. Le rythme régulier de ses coups faisait vibrer la structure de l'espace et du marteau de \carac naquit le temps, première horloge connue du monde, dont les échos résonnaient dans l'immensité du vide. Le météore issu de \Mey, Dristt, dansait parmi les étoiles qui s'allumaient peu à peu. Elle décrivait des courbes gracieuses et voguait avec grâce et légèreté dans l'espace, comme un être qui en renonçant à une partie de lui meme était devenu enfin libre. En passant dans leur voisinage ses trajectoires s'illuminaient d'un majestueuse trainée scintillante reflétant des couleurs nouvelles et resplendissantes.


\subsection{\Ogo}

Les œufs de \Tot survécurent à l'\Extinction et parmi l'un d'eux \Ogo avait réussi à se cacher.