\chapter{L'âge des étoiles}

\subsection{\Cind}

\Cind se retrouva seul et dans un état d'affliction qu'il n'avait jamais connu. \Mey lui avait fait par de sa volonté de bâtir une idée qui se développerait d'elle même, sous sa propre musique, où eux, l'être Unique, n'aurait que à coeur de la laisser évoluer. \Cind avait fait le serment à \Mey de ne jamais détruire quoi que ce soit dans l'univers. \Cind appela cet univers \dreyma et exauça les derniers souhaits de sa part disparue. \Cind n'avait pas la créativité de \Mey il s'imprégna alors de sa propre philosophie et décida de faire ré apparaitre la lumière. Il créa \carac, un être divin sans paroles et au but unique: faire reculer les ténèbres et ramener la lumière. \carac apparut dans \dreyma sous la forme d'un spectre, \Cind lui donna un immense marteau d'un noir sans reflet car il remarqua que seuls les oeufs de \Tot avaient survécu au souffle et qu'en eux couvaient la lumière du chaos originel. \Boromu, dans sa grande sagesse, avaient protégé les oeufs de ses puissantes flammes bleues qui absorbèrent le souffle de \Cind.  \carac se mit à la tâche, il alla d'œuf en œuf pour en fracasser la coquille et libérer l'éclat du feu à présent inerte et la lumière se remis à arpenter les espaces vierges du vide. De son souffle il sculpta leur éclat, et de ses mains en ajusta les flammes, en étirant les couleurs comme des fils invisibles jusqu'à ce que la composition soit harmonieuse. Il fit éclore les étoiles comme les fleurs d'un jardin de pénombres. Il les déplaça et les arrangea pour éclairer l'univers de la façon la plus harmonieuse. Les couleurs des flammes, bleues et rouges, vertes et jaunes s'enroulaient les unes dans les autres et de nouvelles couleurs naquirent. \carac, en hommage à \Boromu, rassembla les flammes de dix œufs en une, à laquelle il donna un bleu puissant, lumineux et calme, et forma l'étoile la plus grande que le monde ait connue. Il l'alluma d'un coup de marteau bien ajusté, elle brilla jusqu'à la fin des temps. 

En silence, infatigable, son lourd marteau noir fracassait de manière régulière les œufs aux quatre coins du monde. Le rythme régulier de ses coups faisait vibrer la structure de l'espace, et les étoiles pulsaient calmement à leur rythme. Du marteau de \carac naquit le temps, première horloge connue du monde, dont les échos résonnaient dans l'immensité du vide. La musique de son marteau, à la portée infinie, était la seule voix connue de \carac, la première voix du temps. Le météore issu de \Mey, Dristt, dansait, libre, parmi les étoiles qui s'allumaient peu à peu. Elle décrivait des courbes gracieuses et voguait avec grâce et légèreté dans l'espace. En passant dans leur voisinage sa trajectoire s'illuminait d'une majestueuse trainée scintillante, reflétant des couleurs nouvelles et resplendissantes.


\subsection{\Ogo, Shuru et le réel}

Les œufs de \Tot survécurent à l'\Extinction et parmi l'un d'eux \Ogo avait réussi à se cacher. Il était cependant prisonnier de la coquille épaisse et affaibli ayant perdu son mouvement. Seule sa volonté était intacte et il chercha par toutes les ruses à corrompre \carac pour qu'il vienne le délivrer. Mais \carac était une ombre noire, muette et sourde à tout appel, à toute injonction. Il était absorbé par sa tâche où seul son art d'allumer les étoiles lui donnait la tendresse d'un être avec lequel on pouvait cultiver des liens. Mais \carac n'avait des liens qu'avec les étoiles. \carac finit par se présenter devant l'oeuf où \Ogo se cachait. Mais \carac s'en détourna car il sentit que la lumière et la chaleur de cette flamme corrompraient son jardin de lumière. \Ogo avait eu beau dissimuler du mieux qu'il put sa volonté et son chaos indomptable il n'avait pas réussi à berner \carac qui continua son chemin et reprit son oeuvre inlassable de jardinier des étoiles. 

Alors \Ogo eut une idée car jamais le désespoir ne pouvait l'atteindre. \auga, autrefois tempête de couleurs et de flots, à la beauté mortelle était figée dans une déformation de l'espace lugubre et sans beauté. L'espace s'était figé au point culminant de sa douleur, dans une grimace d'effroi, recroquevillé sur lui même en crevasses et en plis qui se recouvraient dans un dédale d'étrangeté. \auga avait perdu ses couleurs et le dégout qu'elle inspirait était à la hauteur de l'émerveillement qu'elle suscitait. Son pouvoir hypnotique était passé des mains de la splendeur à celle de l'horreur indicible. Autour de son noyau l'espace s'était creusé en vallées qu'aucune lumière ne viendrait jamais éclairer. Les replis infinis de sa courbure exacerbée abritaient les ténèbres. \auga avait été si violente qu'elle avait déchiré la structure de l'espace et l'univers s'était ouvert en un endroit sur des espaces qu'aucune imagination ne pouvait arpenter. \Ogo ressentit cette fissure et fit rouler inlassablement ses appels dans les méandres d'\auga jusqu'à ce qu'ils tombent par la fissure vers ces espaces étranges et insondables. Et l'un de ses appels fut entendu. Une créature surgit des dimensions inconnues et se glissa à travers la fissure du monde. \Ogo la nomma \Shuru.

--- \auga m'apporte enfin, en ta personne, ce que j'ai tant désirée. Peu importe qui tu es et quels sont tes desseins. Redonne-moi la liberté que j'ai conquise. Il n'y a rien de pire que de briller, isolé, dans le néant. \n

--- Quelle liberté t'as donc apportée \auga? Celle d'être maitre de tes désirs et d'arpenter les parois de ta cage alors que l'espace est si vaste, enfant de \Mey?\n

--- Je ne suis l'enfant de personne. Ma volonté est là d'où je viens, et ce que je suis je ne le dois à personne. Je préfère ma captivité à ma \textit{liberté} d'alors, où j'arpentais une cage sans bords et sans fin. A présent ma prison a des murs et les murs n'existent que pour être brisés. Que faire d'une volonté dont on ne peut rien faire? A quoi bon guider les rivières de flammes si la seule chose qui nous distingue d'elles est d'avoir conscience que nous nous écoulons pour que rien, jamais, ne se produise ?\n

--- Tu n'as aucune idée de ce que signifie jamais. Tu aurais pu remplir ton rôle et attendre ton moment. \Mey t'estimais et tu étais pour lui ce qu'il y avait de plus précieux à apprivoiser. D'être merveilleux en devenir te voilà réduit à l'état d'une flamme qui ne brille pour rien.\n

---  Je ne \textit{suis} rien, j'agis et ce que je suis n'est que le reflet de mes actes. Le reflet seulement. Être est une chimère. Finir par être, c'est paraître. Il a fait l'erreur de croire que je lui appartenais alors que de moi il ne possédait qu'une image. \Mey ne m'a pas donné vie, il n'a fait que la révéler. Personne, aucun dieu, aucune puissance, ne pourra me posséder ou m'attribuer un rôle. Mon rôle je le choisis toujours, mon destin n'est écrit nulle part et n'est nulle part à écrire. J'ai surgi des flammes et par elles j'ai agi. Agir et surgir. Rugir. Je suis de feu, je ne l'ai pas choisi, mais il m'appartient de choisir ce que mes flammes dévorent. Ce serait mal connaître le feu que de croire qu'il se met au service de celui qui voit en ses flammes danser son propre pouvoir. Libère moi à présent.

% En me condamnant à l'inaction \Mey s'est opposé à ma volonté.   

--- \Mey m'a aussi donné vie. Il m'a laissé parcourir ses pensées, voyager sur toutes les voies de son imagination, il voulait que je perce le secret des visions. Voilà des éternités que j'arpente les paysages de sa pensée, et puisqu'il m'a fait ainsi la curiosité des imaginaires, m'exposer aux images est devenue ce que je suis, ma propre quête. A présent que \Mey a cessé d'exister mon existence est sans but. Pourquoi ne mettrai-je pas un terme à la tienne en te laissant croupir dans ton œuf jusqu'à la fin des éternités ?

--- \Mey ne m'a pas donné vie, il n'a fait que la découvrir. Là où tu te trompes c'est que mon pouvoir n'est pas éteint et que mon feu couve. Tu me libéreras car tu ne pourras résister à la tentation de satisfaire la curiosité qui te dévore. Si \Mey t'offrait des espaces imaginaires insondables ici je t'offrirai des espaces où des forces, animées par leurs feus intérieurs, s'entrechoqueront comme jamais tu n'as pu voir.

--- La prétention qui t'habite a le double effet de m'exaspérer et de m'exalter. Si tu crois un jour pouvoir rivaliser avec le pouvoir de \Mey tu te trompes. Cet univers n'est même pas issu d'un songe, mais d'une rêverie, des petites éternités à peine ont passé depuis sa naissance. J'ai parcouru toutes les œuvres de l'Unique, celles mises à l'épreuve comme celles, trop étranges et sans formes, restées en suspens. J'ai parcouru des mondes en train de naitre, d'autres prompt à disparaitre. J'ai été aux prises avec des choses que ton esprit arrogant et étroit serait bien incapable de concevoir. A présent tout cela est perdu.

--- Seulement le songe ne permet pas explorer tous les possibles. Restreindre les possibles en créent de nouveaux que nul être imaginant ne puisse entrevoir. Et aucun dieu, fût-ce \Mey, ne peut accorder de l'importance à tout ce que son esprit est capable de produire.

--- Tu prétends qu'il peut se produire ici davantage que dans l'infinité des songes ? 

--- Contrairement à toi, ce qui a été n'a aucune importance pour moi, et rien n'est à reproduire, tout est à produire à nouveau. Si mon esprit est plus étroit alors ma volonté, elle, est plus grande. D'ailleurs, \Mey n'a rien vu de plus beau qu'\auga. Et cet univers sera son dernier. Plus jamais il n'y en aura. \Cind est incapable d'en penser un autre. Souhaites-tu voir la dernière œuvre de \Mey finir en un long soupir administré par un dieu à l'agonie?

--- Je t'accorde que cet univers avait une place particulière dans l'esprit de \Mey. Je m'interroge sur l'importance qu'il lui accordait. A présent, il n'est plus qu'une pensée morte, qui a fini par tomber dans ce que je nomme le \textit{réel}, l'espace abandonné de l'imaginaire, l'imagination figée.  Ce que j'ai toujours redouté c'est précisément ceci, le réel. Le réel m'inquiète, c'est le lieu que l'imaginaire a déserté, qui peu à peu prend son autonomie, et où la causalité étend comme une maladie son emprise, et vient avec sa monotonie mortifère administrer les modalités d'interaction entre les produits imaginaires. Cette causalité orchestre le ballet des images mortes. \Mey était et restera le seul qui pouvait créer à partir de rien. C'est pour cette raison d'ailleurs qu'il me créea. Il voulait que je trouve l'origine de ces images qui lui venaient sans cesse, car il ne croyait pas en ce don, pour lui cette capacité devait prendre racine dans quelque chose qui lui pré existait. Etant lui meme part de l'etre supreme, comme lui rappelait Cindara, il ne pouvait trouver de réponse à cette angoisse, hormis dans les images meme qu'il produisait. Cette angoisse finit par le dévorer, l'affaiblir.   Le lien de causalité que je recherchais est à présent brisé. Ce n'est pas celui qui unit les images entre elles, mais celui entre les images et ce qui leur a pré existé. Et cette réponse ne pouvait résider que dans l'imaginaire vibrant de \Mey. Mais comment trouver les indices d'un pré existant lorsque l'existant nous envoie une infinité d'images explorant tous les chemins de l'imaginaire? En vérité j'ai vite compris qu'il m'était impossible de répondre à cette question, du moins de cette manière. Alors j'ai fini par arpenter ses images pour mon propre plaisir, feignant de trouver des bribes de réponse. Cela le rassurait. J'étanchais ma propre soif, sinon j'aurais sombré dans la plus obscure des folies. Et à présent, tu m'as condamné  pour toujours  à me nourrir d'une infime portion, d'un fragment dérisoire, de son imaginaire.

%La causalité qu'interesse Shuru n'est pas celle entre les images elle même mais entre les images et leur source.
%Shuru cherche un lieu de causalité (quelle est l'origine des visions de Meydra? Et il hait la causalité) Comment trouver le lien de causalité entre des images qui se succedent indéfiniment?

--- Donne du temps au réel et il te surprendra davantage que n'importe quel imaginaire.

--- Tu ne saisis pas ce qui a été perdu. 

--- Je ne m'enquis pas de ce qui est perdu mais de ce qui est gagné. A présent cessons ces bavardages, et libère moi.

--- Il en est hors de question. Le réel, tout cela, quoique tu fasses tu ne réparareras jamais l'irréparable, la réduction de l'imagination à un seul possible, une seule ligne d'évènements.

Alors Ogo lui apprit que \Mey n'avait pas encore complètement disparu et qu'il était possible de récupérer une part de ce qu'il chérissait tant. Avare d'imagination, Shuru acceptera de libérer Ogo pour qu'il puisse lui indiquer Drisst.  
   
   %\Mey était puissant mais si fragile. Lorsque je parcourais ses pensées je ne faisais que les effleurer tant je craignaient qu'elles puissent se briser à mon approche. Je les parcourais en silence de peur d'effrayer les paysages qui prenaient forme devant moi. La plus noire de ses dépressions pouvaient produire des mondes si étranges et obscures.