\chapter{L'âge du chaos}


\section{\Mey et \Cind}


\subsection{Les \SC}

L'univers est née de l'imagination de \Mey, l'un des deux visages de l'Unique, au cours d'une rêverie. \Cind, le deuxième visage, pour préserver cette idée à la fois forte et fragile, l'embrasa tant qu'elle demeurait incertaine, afin que rien ne puisse y demeurer ou y être créée. L'univers n'était alors qu'une fournaise, parcourue de flammes, où seuls régnaient la chaleur ardente et le chaos. Cependant cela ne convenait pas à \Mey, car le chaos agitait ses rêves et ses pensées. Cet univers commençait à accaparer son esprit tout entier, il absorbait toutes ses volontés. Cette pensée crût si fort en lui que le chaos perpétuel finit par le menacer. Pour y remédier, \Cind se mit alors à imaginer des courants de flammes et des rivières de feu. Ces mouvements cohérents et coordonnés, cette danse cosmique, donnèrent à l'univers une structure encore grossière et instable. Les courants de feu avaient chacun leur propre dynamique, leur propre individualité: certains d'entre eux étaient vifs et rougeoyants, d'autres, immenses courants de flammes bleues et sourdes, s'écoulaient sans fin, imperturbables, et avec davantage de lenteur.

\Mey fût apaisé en ressentant l'écoulement des rivières de feu et il commença à les considérer comme des êtres à part entière et à les chérir comme ses propres enfants. Finalement \Mey leur donna vie sous la forme de serpents cosmiques, les \SC. Nul n'aurait alors pu imaginer leur taille, car elle n'était comparable à rien, hormis à celle de l'univers, et aux limites de l'esprit de \Mey qu'aucun être ne peut concevoir ou se représenter. Les \SC parcouraient indéfiniment l'univers, se repoussaient lorsqu'ils étaient trop proches les uns des autres et s'attiraient lorsqu'ils étaient trop éloignés ou isolés. Leurs mouvements contrôlaient les tempêtes de feu qui déchiraient l'espace. Ils étaient alors les gardiens de l'équilibre de l'imaginaire de \Mey, et de l'existence de l'univers.

\subsection{Naissance des \Dormus}

\Mey développa une affection particulière pour quatre d'entre eux. Il leur donna davantage de pouvoir en les dotant d'une volonté propre et d'un libre-arbitre. \Cind était opposé à cette idée, il la trouvait dangereuse et s'inquiétait toujours plus pour \Mey. Il savait que les \Dormus menaceraient l'équilibre du monde en se détournant de la tâche qu'il leur avait assignée à l'origine. Il craignait que la chaos puisse revenir. \Mey avait de nobles intentions mais il supportait mal les conséquences de ses actes. \Cind avait à l'origine embrasé le monde pour permettre à \Mey de cerner ses aspirations les plus grandes, mais aussi pour écarter celles qui lui semblaient les plus dangereuses, car \Cind était lié à \Mey, à ses souffrances, à ses joies et à son existence.

Les \Dormus avaient envouté \Mey. \Boromu était le plus immense et le plus brûlant des \SC, il était d'un bleu lumineux parsemé de reflets argentés. D'immenses éclairs silencieux irisaient parfois l'intérieur de son corps. Il était le plus sage des \Dormus, et ne se départit jamais de son rôle, il allait inlassablement, avec lenteur, apaisant toutes les tempêtes de feu qui se déclaraient, repassant les tumultes de son immense flot bleu et laissant derrière lui des écoulements de flammes laminaires et pacifiés. Dans sa tâche il fût aidé par \Esu, le plus majestueux d'entre eux. Plus petit que \Boromu, il parcourait l'univers avec hâte et aisance. Ses couleurs étaient les plus belles, en lui les flammes jouaient harmonieusement de la palette du chaos tout entier. En ce sens il incarnait le chaos originel maitrisé, il était capable de ressentir ses moindres soubresauts,frémissements ou anomalies. Il était l'oreille de l'univers et le berger des \SC. \Esu était le plus parfait des \Dormus. Telle était à la fois sa force et sa faiblesse. Il remplissait son rôle mieux que quiconque, mais pourtant cela ne lui apportait aucun plaisir et aucune satisfaction. Majestueux mais sans désirs propres, sa place ne pouvait que lui être offerte par la reconnaissance et l'amour que \Mey lui témoignaient. A cette période, le chaos était apprivoisé comme jamais il ne le fût, \Mey n'eut jamais à en souffrir. Mais \Mey, bien qu'il chérissait \Esu comme les autres, jetait sa préférence sur \Ogo, le plus turbulent et le plus créatif des quatre. \Esu, le plus parfait des \Dormus, sombrait dans le malheur.

\Ogo était le plus vif des \Dormus, à la volonté la plus forte. Il était d'une intelligence et d'une ruse remarquables. Il brûlait d'un tourbillon de rouge et de vert, en lui-même se déchainaient des tempêtes et il conservait en lui le chaos originel, indompté et tempétueux. Parmi les \Dormus, il était le seul à saisir le potentiel de la vie qui lui avait été donnée. Il méprisait \Boromu, et il haïssait \Tot. \Tot, le quatrième, brûlait d'un blanc pur et aveuglant. Il avait immédiatement redouté son existence en tant qu'elle impliquait irrémédiablement sa propre disparition. Il avait fait de cette dialectique le cœur même de son existence et il fût le seul à détenir la capacité de donner vie à des \SC. Les œufs qu'il disséminait sur son passage, et dans lesquels couvaient son feu et une part de lui-même, apaisaient l'angoisse de sa disparition. Pour \Ogo, dont les désirs fleurissaient inlassablement, \Tot
était la vie sous sa forme la plus pathétique et méprisable. 

%\Ogo supportait mal ses congénères, il pensait qu'il devait être l'unique être à posséder une volonté.

\subsection{La duperie d'\Ogo et l'\Extinction}


 \Ogo demanda à \Mey davantage de pouvoir. \Mey, sur les conseils de \Cind, rejeta ses demandes et lui enjoignit la patience. Mais \Ogo brûlait d'un désir si grand qu'il vécut le don de volonté sans moyens de la réaliser comme une humiliation et grandit en lui une rancœur que jamais aucun souffle ne pourra éteindre.  
 
 
  Ses demandes laissées sans réponse lui apprirent peu à peu le moyen d'atteindre \Mey et de le contraindre à accepter de lui donner ce qu'il lui revenait. \Ogo alla voir \Esu, dont il connaissait la faiblesse. \Ogo dit à \Esu que la reconnaissance de \Mey ne pouvait lui être définitivement témoignée que s'il se distinguait dans la réalisation d'un acte unique. Pour cela, il lui demanda de lui indiquer le point du chaos le plus instable que sa sensibilité pouvait résoudre parmi l'immensité des courants de feu. Il en attiserait le mouvement et les flammes jusqu'à ce que cela fasse souffrir \Mey suffisamment. Alors \Esu pourrait venir éteindre la tempête et attirer ainsi l'intention de \Mey et gagner définitivement sa reconnaissance. \Esu, désespéré, se laissa convaincre et indiqua à \Ogo ce point du chaos. Il s'y rendit pendant des éternités, s'écoulant sans fin parmi les \SC. Il repéra alors le point du chaos où des vortex de flammes bleues, blanches et rouges s'affrontaient, s'aspiraient et se rejetaient sans cesse dans de grands fracas et d'ondes de chocs qui faisaient vibrer et disloquaient les \SC. \Ogo se joignit aux tourbillons, et tout en accélérant les mouvements de flammes en créa de nouveaux à l'intérieur des plus grands, il força les \SC à suivre des trajectoires tortueuses et à amplifier la circulation des courants de flammes. La tempête finit par prendre une telle ampleur que \Mey s'affaiblit, il fut à la fois subjugué par sa beauté, car rien encore dans cet univers ne fut aussi beau et unique, et terrassé en lui même par la douleur qu'elle lui causait. Alors \Esu finit par s'y rendre, mais lorsqu' il arriva la tempête était si violente et si forte qu'il ne put rien faire pour la calmer par ses propres moyens. La tempête de chaos se composaient de tant de tourbillons entrelacés, ses mouvements entrainaient tous les courants de flammes dans sa danse et elle ne cessait de s'accroitre, plus elle dévorait de \SC et plus elle gagnait en violence et enflait davantage. \Esu se résigna et finit par demander de l'aide à \Boromu, mais des éternités entières se seraient écoulées avant qu'il puisse l'atteindre et la tempête aurait pris alors tellement d'ampleur qu'elle aurait aspiré \Boromu, le plus grands des \Dormus, pour en faire une flamme bleue de l'un de ses bras dévorants l'espace. \Boromu le savait et il comprit à la fois tous les évènements à l'origine de cette tempête mais également toutes les conséquences qu'elle aurait. Au lieu d'écouter les appels à l'aide d'\Esu, qui pathétique, maudissait \Ogo et s'apitoyait sur son propre sort. \Boromu réunissa tous les œufs de \Tot qu'il rencontrait et les engloutit. \Tot, aussitôt qu'il eut pris connaissance de l'existence de la tempête s'était enfui aux confins de l'univers et aucun \Dormus ne le revit jamais.      
   
  
  \Mey malgré la douleur, était hypnotisé par la tempête. Elle avait une forme si particulière, elle dansait dans son esprit, tumultueuse et enchanteresse, elle lui donnait à voir des couleurs et des motifs nouveaux dans des associations qu'il n'aurait jamais pu imaginer lui-même. C'était une source d'inspiration et la détruire lui était impossible car il n'avait jamais rien vu de si beau, il l'appela \auga. Mais \auga à force de croitre finit par menacer \Mey et le détruire complètement. \Cind, intervint et il souffla de toutes ses forces sur l'univers et anéantit tout mouvement: les \SC et les flammes, les \Dormus et la tempête, qui se figea, cristallisée dans sa forme et sa structure étranges et fractale, et disparut dans les ténèbres. \Mey fut sauvé mais il en voulut malgré tout tellement à \Cind, que l'être unique se sépara en deux êtres distincts. Malgré tout, \Mey était en train de disparaitre et il ne survivrait pas à cette séparation. Il sortit l'univers de ses pensées. Il fit part à \Cind de ses derniers souhaits et de ses intentions. \Mey, l'être unique, las de son éternité, quitta sa propre dimension et avant de disparaitre une partie de lui se projeta dans son univers, sa dernière idée, sous la forme d'un météore.
  
  
  
 % Meydra en danger. Cindara fera tout pour l’empecher en anéantissant les flammes, en refroidissant l’univers et en essayant de tuer les Omu [Donner un nom a cet evenement] Certains survivront (voir au dessus). Meydra malgré sa propre mise en danger de disparition en voudra terriblement à Cindara pour son geste, et l’être unique se séparera en 2 etres distincts. Finalement Meydra, apres plusieurs soubresaults et un dernier coup de création de génie ( a trouver) finira par disparaître en faisant part à Cindara de ses intentions quant à cet univers (quelles étaient elles ? Peut etre de pour une fois créer quelque chose, un monde, qu’il ne chercherait pas a controler mais qui ferait par lui meme des choses plus merveilleuses que lui, en préparant juste la terre et y laisser pousser les graines. Ou alors décider de ne faire que créer et ne rien détruire soi meme, interdiction de tuer quoi que ce soit, juste ajouter ne jamais rien enlever) et seul restera Cindara, qui fera tout pour continuer l’œuvre de Meydra en essayant de batir un univers harmonieux en l’honneur de Meydra qui était toujours dévoré par ses créations. 
  