\chapter{L'âge du chaos}


\section{\Mey et \Cind}

L'univers est née de l'imagination de \Mey, l'un des deux visages de l'Unique, au cours d'une rêverie. \Cind, le deuxième visage, pour préserver cette idée à la fois forte et fragile, l'embrasa tant qu'elle demeurait incertaine, afin que rien ne puisse y demeurer ou y être créée. L'univers n'était alors qu'une fournaise, parcourue de flammes, où seuls régnaient la chaleur ardente et le chaos. Cependant cela ne convenait pas à \Mey, car le chaos agitait ses rêves et ses pensées. Cet univers commençait à accaparer son esprit tout entier, il absorbait toutes ses volontés. Cette pensée crût si fort en lui que le chaos perpétuel finit par le menacer. Pour y remédier, \Cind se mit alors à imaginer des courants de flammes et des rivières de feu. Ces mouvements cohérents et coordonnés, cette danse cosmique, donnèrent à l'univers une structure encore grossière et instable. Les courants de feu avaient chacun leur propre dynamique, leur propre individualité: certains d'entre eux étaient vifs et rougeoyants, d'autres, immenses courants de flammes bleues et sourdes, s'écoulaient sans fin, imperturbables, et avec davantage de lenteur.

\Mey fût apaisé en ressentant l'écoulement des rivières de feu et il commença à les considérer comme des êtres à part entière et à les chérir comme ses propres enfants. Finalement \Mey leur donna vie sous la forme de serpents cosmiques, les \SerpentsCosmiques. Nul n'aurait alors pu imaginer leur taille, car elle n'était comparable à rien, hormis à celle de l'univers, et aux limites de l'esprit de \Mey qu'aucun être ne peut concevoir ou se représenter. Les \SerpentsCosmiques parcouraient indéfiniment l'univers, se repoussaient lorsqu'ils étaient trop proches les uns des autres et s'attiraient lorsqu'ils étaient trop éloignés ou isolés. Leurs mouvements contrôlaient les tempêtes de feu qui déchiraient l'espace. Ils étaient alors les gardiens de l'équilibre de l'imaginaire de \Mey, et de l'existence de l'univers.

\Mey développa une affection particulière pour quatre d'entre eux. Il leur donna davantage de pouvoir en les dotant d'une volonté propre et d'un libre-arbitre. \Cind était opposé à cette idée, il la trouvait dangereuse et s'inquiétait toujours plus pour \Mey. Il savait que les \Dormus menaceraient l'équilibre du monde en se détournant de la tâche qu'il leur avait assignée à l'origine. Il craignait que la chaos puisse revenir. \Mey avait de nobles intentions mais il supportait mal les conséquences de ses actes. \Cind avait à l'origine embrasé le monde pour permettre à \Mey de cerner ses aspirations les plus grandes, mais aussi pour écarter celles qui lui semblaient les plus dangereuses, car \Cind était lié à \Mey, à ses souffrances, à ses joies et à son existence.

Les \Dormus avaient envouté \Mey. \Boromu était le plus immense et le plus brûlant des \SerpentsCosmiques, il était d'un bleu lumineux parsemé de reflets argentés. D'immenses éclairs silencieux irisaient parfois l'intérieur de son corps. Il était le plus sage des \Dormus, et ne se départit jamais de son rôle, il allait inlassablement, avec lenteur, apaisant toutes les tempêtes de feu qui se déclaraient, repassant les tumultes de son immense flot bleu et laissant derrière lui des écoulements de flammes laminaires et pacifiés.
