\chapter{L'âge du chaos}


\section{\Mey et \Cind}

L'univers est née de l'imagination de \Mey, l'un des deux visages de l'Unique, au cours d'une rêverie. \Cind, le deuxième visage, pour préserver cette idée à la fois forte et fragile, l'embrasa tant qu'elle demeurait incertaine, afin que rien ne puisse y demeurer ou y être créée. L'univers n'était alors qu'une fournaise, parcourue de flammes, où seuls régnaient la chaleur ardente et le chaos. Cependant cela ne convenait pas à \Mey, car le chaos agitait ses rêves et ses pensées. Cet univers, toujours plus ancré dans son esprit, courtisait toutes ses volontés. Cette pensée crût si fort en lui que le chaos perpétuel le menaça. Pour y remédier, \Cind se mit alors à imaginer des courants de flammes et des rivières de feu. Ces mouvements cohérents et coordonnés, cette danse cosmique, donnèrent à l'univers une structure encore grossière et instable. Les courants de feu avaient chacun leur propre dynamique, leur propre individualité: certains d'entre eux étaient vifs et rougeoyants, d'autres, immenses courants de flammes bleues et sourdes, s'écoulaient sans fin, imperturbables, et avec davantage de lenteur. \Mey fût apaisé et il commença à les chérir comme ses propres enfants. Finalement \Mey leur donna vie sous la forme de serpents cosmiques, les \SerpentsCosmiques. Nul n'aurait alors pu imaginer leur taille, car elle n'était comparable à rien, hormis à celle de l'univers, et aux limites de l'esprit de \Mey qu'aucun être ne peut concevoir ou se représenter. Les \SerpentsCosmiques parcouraient indéfiniment l'univers, se repoussaient lorsqu'ils étaient trop proches les uns des autres et s'attiraient lorsqu'ils étaient trop éloignés ou isolés. Leurs mouvements contrôlaient les tempêtes de feu qui déchiraient l'espace. Ils étaient alors les gardiens de l'équilibre de l'imaginaire de \Mey, et de l'existence de l'univers.

,  

