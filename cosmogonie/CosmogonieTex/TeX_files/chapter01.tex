\chapter{L'âge du chaos}


\section{\Mey et \Cind}

L’univers n’est que flammes, chaleur ardente et feu. Une fournaise. L’univers a été crée par deux êtres transcendants : il est née de l’imagination de \Mey, en attendant de savoir ce qu’ils allaient en/y faire \Cind l’a embrasé pour que rien ne puisse y être créée, que rien n’y dure. 

Cependant cela ne convient pas a \Mey, car le chaos règne, cela agite ses rêves et ses pensées. Cet univers est à présent trop ancré dans sa volonté, dans son esprit, il y tient, cette pensée croit en lui, et le chaos perpétuel le menace. Petit à petit, pour y remédier, il se met à imaginer des courants de flammes, des mouvements coordonnés, des choses s’animent et tempèrent le chaos en lui donnant une structure grossière. Un ordre émerge même si c’est toujours une fournaise, même si tout brûle, des rivières de feu apparaissent. Il imagine ces courants comme des serpents cosmiques. C’est la naissance des serpents de feu cosmiques, les \SerpentsCosmiques. On ne connaît pas leur taille car on ne peut la comparer avec rien, hormis avec la taille de l’univers, aux limites de l’esprit de \Mey. Certains sont plus petits que d’autres, d’autres font peut-être la taille de l’univers lui même, si cette taille est mesurable. Elle n’est pas imaginable, hormis dans l’esprit des êtres suprêmes.

Les \SerpentsCosmiques apaisent \Mey et il commence à les chérir comme ses propres enfants. Il commence à les imaginer comme des êtres fait de fournaise mais évoluant dedans. Leur mouvement assure l’homogénéité de la fournaise, ils se repoussent lorsqu’ils sont trop près les uns des autres et s’attirent lorsqu’ils sont trop éloignés ou isolés. Ils assurent l’équilibre de l’imaginaire de \Mey et l’existence même de l’univers. Petit à petit \Mey leur donne vie en imaginant ces courants de flammes qui lui deviennent de plus en plus familiers, uniques, avec leurs propres caractéristiques. Les courants ne se mélangent pas, ils ont chacun leur propre dynamique, leur propre mouvement.

