\chapter{L'âge du chaos}


\section{\Mey et \Cind}

L'univers est née de l'imagination de \Mey, l'un des deux visages de l'Unique, au cours d'une rêverie. \Cind, le deuxième visage, pour préserver cette idée à la fois forte et fragile, l'embrasa tant qu'elle demeurait incertaine, afin que rien ne puisse y demeurer ou y être créée. L'univers n'était alors qu'une fournaise, parcourue de flammes, où seuls régnaient la chaleur ardente et le chaos. Cependant cela ne convenait pas à \Mey, car le chaos agitait ses rêves et ses pensées. Cet univers commençait à accaparer son esprit tout entier, il absorbait toutes ses volontés. Cette pensée crût si fort en lui que le chaos perpétuel finit par le menacer. Pour y remédier, \Cind se mit alors à imaginer des courants de flammes et des rivières de feu. Ces mouvements cohérents et coordonnés, cette danse cosmique, donnèrent à l'univers une structure encore grossière et instable. Les courants de feu avaient chacun leur propre dynamique, leur propre individualité: certains d'entre eux étaient vifs et rougeoyants, d'autres, immenses courants de flammes bleues et sourdes, s'écoulaient sans fin, imperturbables, et avec davantage de lenteur.

\Mey fût apaisé en ressentant l'écoulement des rivières de feu et il commença à les considérer comme des êtres à part entière et à les chérir comme ses propres enfants. Finalement \Mey leur donna vie sous la forme de serpents cosmiques, les \SerpentsCosmiques. Nul n'aurait alors pu imaginer leur taille, car elle n'était comparable à rien, hormis à celle de l'univers, et aux limites de l'esprit de \Mey qu'aucun être ne peut concevoir ou se représenter. Les \SerpentsCosmiques parcouraient indéfiniment l'univers, se repoussaient lorsqu'ils étaient trop proches les uns des autres et s'attiraient lorsqu'ils étaient trop éloignés ou isolés. Leurs mouvements contrôlaient les tempêtes de feu qui déchiraient l'espace. Ils étaient alors les gardiens de l'équilibre de l'imaginaire de \Mey, et de l'existence de l'univers.

\Mey développa une affection particulière pour quatre d'entre eux. Il leur donna davantage de pouvoir en les dotant d'une volonté propre et d'un libre-arbitre. \Cind était opposé à cette idée, il la trouvait dangereuse et s'inquiétait toujours plus pour \Mey. Il savait que les \Dormus menaceraient l'équilibre du monde en se détournant de la tâche qu'il leur avait assignée à l'origine. Il craignait que la chaos puisse revenir. \Mey avait de nobles intentions mais il supportait mal les conséquences de ses actes. \Cind avait à l'origine embrasé le monde pour permettre à \Mey de cerner ses aspirations les plus grandes, mais aussi pour écarter celles qui lui semblaient les plus dangereuses, car \Cind était lié à \Mey, à ses souffrances, à ses joies et à son existence.

Les \Dormus avaient envouté \Mey. \Boromu était le plus immense et le plus brûlant des \SerpentsCosmiques, il était d'un bleu lumineux parsemé de reflets argentés. D'immenses éclairs silencieux irisaient parfois l'intérieur de son corps. Il était le plus sage des \Dormus, et ne se départit jamais de son rôle, il allait inlassablement, avec lenteur, apaisant toutes les tempêtes de feu qui se déclaraient, repassant les tumultes de son immense flot bleu et laissant derrière lui des écoulements de flammes laminaires et pacifiés. Dans sa tâche il fût aidé par \Esu, le plus majestueux d'entre eux. Plus petit que \Boromu, il parcourait l'univers avec hâte et aisance. Ses couleurs étaient les plus belles, en lui les flammes jouaient harmonieusement de la palette du chaos tout entier. En ce sens il incarnait le chaos originel maitrisé, il était capable de ressentir ses moindres soubresauts,frémissements ou anomalies. Il était l'oreille de l'univers et le berger des \SerpentsCosmiques. \Esu était le plus parfait des \Dormus. Telle était à la fois sa force et sa faiblesse. Il remplissait son rôle mieux que quiconque, mais pourtant cela ne lui apportait aucun plaisir et aucune satisfaction. Étant dénué d'une forte volonté comme les autres, conscient de sa majesté, ses propres désirs ne pouvaient éclore que dans la reconnaissance que lui accordait \Mey. Il cherchait désespérément à être aimé et reconnu de \Mey, pour trouver sa propre place. Tous ses actes étaient accomplis non pas pour se réaliser lui-même mais pour \Mey. A cette période, le chaos était apprivoisé comme jamais il ne le fût, \Mey n'eut jamais à en souffrir. Mais \Mey, bien qu'il chérissait \Esu comme les autres, jetait sa préférence sur \Ogo, le plus turbulent des quatre. \Esu sombra dans le malheur.

\Ogo était le plus vif des \Dormus, aux volontés les plus fortes. Il était d'une grande intelligence et sa ruse lui permettait de nourrir ses propres desseins. Il brûlait d'un tourbillon de rouge et de vert, en lui même se déroulaient des tempêtes et il conservait en lui le chaos originel, indompté et tempétueux. Parmi les \Dormus, il était le seul à saisir le potentiel de la vie qui lui avait été donnée. Il méprisait \Boromu, et il haïssait \Tot. \Tot, le quatrième, brûlait d'un blanc pur et aveuglant. Il avait immédiatement redouté son existence en tant qu'elle impliquait irrémédiablement sa propre disparition. Il avait fait de cette dialectique le cœur même de son existence et il fût le seul à détenir la capacité de donner vie à des \SerpentsCosmiques. Les œufs qu'il disséminait sur son passage, et dans lesquels couvaient son feu et une part de lui-même, apaisaient l'angoisse de sa disparition. Pour \Ogo, dont les désirs fleurissaient inlassablement, \Tot
était la vie sous sa forme la plus pathétique et méprisable. Il demanda rapidement à \Mey davantage de pouvoir. \Mey, sur les conseils de \Cind, refusa ses demandes et lui enjoignit la patience. \Ogo supportait mal ses congénères, il pensait qu'il devait être l'unique être à posséder une volonté. \Ogo ne supportait pas d'être limité dans ses désirs, il rejetait \Mey. Au cours de ses échanges avec \Mey, il comprit peu à peu sa fragilité. \Ogo alla voir \Esu, dont il connaissait ses faiblesses et il le dupa pour atteindre \Mey. Il s'y prit ainsi.


%Cette faiblesse sera exploitée par Ogo : Esu n’arrive pas a avoir suffisament de reconnaissance de Meydra suffisament (qui estime davantage Ogo pour son potentiel créateur). Il est malheureux, n’obtient pas la reconnaissance qu’il souhaite. Ogo le dupera en lui disant que si Meydra ne le remarque pas c’est parce qu’il éteint tous les points de chaos avant meme qu’ils ne soient une douleur pour Meydra. Il lui propose de prendre le point d’instabilité le plus grand qu’il ressente, que lui Ogo s’y rende pour attiser le feu pour attirer l’attention de Meydra et qu’ensuite il laisserait Eru l’éteindre et ainsi gagner sa reconnaissance. Esu, déséspéré cedera, indiquera Ogo au point, mais il sera trop tard quand lui arrivera sur place, la tempete sera deja trop immense pour qu’il puisse la calmer seul. Il demandera l’aide de Boromu, mais ce dernier, le seul a pouvoir l’eteindre, ne pourra pas arriver a temps, il sera ralenti par Ogo. Esu est immature et infantile, son estime de soi passe par la reconnaissance de Meydra.