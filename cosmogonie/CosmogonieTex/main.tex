%\documentclass[a4paper,14pt]{book}
\documentclass[letterpaper,14pt,twocolumn,openany]{dndbook}


\usepackage[utf8]{inputenc}
\usepackage{graphicx}
\usepackage[T1]{fontenc}
\usepackage[francais]{babel}
\usepackage{xspace}
\usepackage{hang}
\usepackage{lipsum}
\usepackage{listings}

\usepackage{hyperref}


\hypersetup{
	colorlinks=true,
	linkcolor=blue,
	filecolor=magenta,      
	urlcolor=cyan,
}

\usepackage{glossaries}


\makeglossaries


\lstset{%
	basicstyle=\ttfamily,
	language=[LaTeX]{TeX},
}

\makeglossaries

% Fait une entree au glossaire en assignant automatiquement le nom au label (#1)
% Argument #2 est la description
\newcommand{\newentry}[2]
{
	\newglossaryentry{#1}{
		name = {#1},
		description={#2}
	}
}

%A utiliser dans le fichier noms Propres: cree une variable #1 pour le string #2 et une entree au glossaire
% #1: Nom commande pour appeler dans le texte (ex: Dieu) #2 : Nom de l'entité qui apparaitra dans le texte (et sera également en entrée du Glossaire) #3: Description dans le glossaire
%   
\newcommand{\newEntity}[3]
{
	% #1 doit etre utilise pour fabriquer une commande \#1
	%\newcommand{\@namedef{##1}}{\gls{#2}\xspace}
	\expandafter\newcommand\csname #1\endcsname {\gls{#2}\xspace}
	\newentry{#2}{#3}
}

%\newEntity{Cind}{Cindara}{Un des deux êtres divins}

\loadglsentries{glossaire}

\begin{document}

%Change la taille de la font doc
\fontsize{14}{17} \selectfont

\author{P. Schuhmacher}
\title{Cosmogonie}
\date{}
 
 
\frontmatter
%\maketitle
%\tableofcontents

\mainmatter


\chapter{L'âge du chaos}


\section{\Mey et \Cind}


\subsection{Les \SC}


Le monde se trouvait enfoui dans la rémanence du grand rêve de \Mey, l'un des deux visages de l'Unique. Le souvenir monde, devenu pensée tenace, se développa alors en un chaos assourdissant, embrasé de désordre, car la vision de son devenir quelque-part révélé fut aussi merveilleuse que fugace, et sans mot dire disparut, laissant à sa suite la rumeur seule de promesses à peine formulées, sous la forme d'une intense agitation sans mesure, dessinant les contours vibrants du monde encore incertain et fragile. Et le désordre s'enflamma et engloutit le monde.

A un tel chaos \Mey aurait fini par succomber tant il agitait ses rêves et ses pensées, accaparerait son esprit tout entier, et absorbait toutes ses volontés. Pour y remédier, \Cind, d'un grand souffle imaginant, mit en mouvement le chaos en de grands courants de flammes et rivières de feu. Ces mouvements cohérents et coordonnés, cette danse cosmique, donnèrent à l'univers sa première structure, encore grossière et instable. Les courants de feu avaient chacun leur propre dynamique, leur propre individualité: certains d'entre eux étaient vifs et rougeoyants, d'autres, immenses courants de flammes bleues et sourdes, s'écoulaient sans fin, imperturbables, et avec davantage de lenteur.

La fureur dissonante du désordre céda sa place à la sourde rumeur des rivières et à la lente clameur des flots, et \Mey en fût apaisé. Il les ressentit si fort en lui, qu'il se mit à les chérir comme ses propres enfants, et les \SC, les serpents cosmiques lui apparurent. Nul n'aurait alors pu imaginer leur taille, car elle n'était comparable à rien, hormis à celle de l'univers, et aux limites de l'esprit de \Mey qu'aucun être ne peut concevoir ou se représenter. Les \SC parcouraient l'univers sans relâche longtemps avant la signification du temps. Ils se repoussaient lorsqu'ils étaient trop proches les uns des autres et s'attiraient lorsqu'ils étaient trop éloignés ou isolés. Leurs mouvements contrôlaient les tempêtes de feu qui déchiraient l'espace, et de l'équilibre de l'imaginaire de \Mey, et de l'existence même du monde, ils en étaient les inébranlables gardiens.


\subsection{Naissance des \Dormus}

Quatre \SC firent l'objet d'une affection très profonde de \Mey. Il leur donna une liberté plus grande en son esprit, et les dota d'une volonté propre et d'une autonomie. A cette idée \Cind s'opposa, il la trouvait dangereuse et s'inquiétait toujours plus pour \Mey. Il pressentait que les quatre \Dormus menaceraient l'équilibre du monde en se détournant de la tâche qu'il leur avait été assignée. Il craignait que le chaos puisse à nouveau revenir. \Cind avait apaisé le monde pour permettre à \Mey de cerner ses aspirations les plus grandes, mais aussi pour mettre au loin celles qui lui semblaient les plus périlleuses, car \Cind était lié d'existence à \Mey. Mais si \Mey était noble dans ses intentions, il supportait mal les conséquences de ses actes. Et s'il ressentit en son cœur l'opposition de \Cind, celui-ci était déjà en proie à une étreinte plus grande encore. Car aucune mise en garde ne pouvait entraver l'expression imaginaire de ses sentiments les plus forts.      

 \Boromu était le plus immense et le plus brûlant des \SC, il était d'un bleu lumineux parsemé de reflets argentés. De larges éclairs dans le lointain silencieux irisaient l'intérieur de son corps. Il était le plus sage des \Dormus, et ne se départit jamais de son rôle, il allait inlassablement, avec lenteur, apaisant toutes les tempêtes de feu qui se déclaraient, repassant les tumultes de son immense flot bleu et laissant derrière lui des écoulements de flammes laminaires et pacifiés. Dans sa tâche il fût aidé par \Esu, le plus majestueux d'entre eux. Plus petit que \Boromu, il parcourait l'univers avec hâte et aisance. Ses couleurs étaient les plus belles, en lui les flammes jouaient harmonieusement de la palette du chaos tout entier. En ce sens il incarnait le chaos originel maitrisé, il était capable d'en ressentir les moindres soubresauts, frémissements ou anomalies. Il était l'oreille de l'univers et le guide infaillible des \SC. \Esu était le plus parfait des \Dormus. Telle était à la fois sa force et sa faiblesse. Bien qu'il remplissait son rôle sans jamais défaillir, de ses actes aucun plaisir n'en jaillissait. Majestueux mais sans désirs qui lui soient propres, sa propre place il ne pouvait la trouver que dans la reconnaissance et l'amour que \Mey lui témoignaient. Et sous la garde d'\Esu, \Mey n'eut jamais à souffrir, ou craindre de souffrir, du refluement du chaos tant les espaces étaient apprivoisés. Mais \Mey affectionnait \Esu comme les autres, et non davantage. Et si en ces temps reculés la question à \Mey eut été posée de savoir qui en son cœur il préférait, \Esu au grand jamais aurait été le nom qu'il prononcerait. Car à présent nous le savons c'est bien d'\Ogo qu'il s'agissait, le plus turbulent et le plus créatif d'entre les quatre. Quant à \Esu, le plus parfait des \Dormus, il sombrait dans le malheur. 

\Ogo était le plus vif des \Dormus, à la volonté la plus forte. Il était d'une intelligence et d'une ruse remarquables. Il brûlait d'un tourbillon de rouge et de vert, en lui-même se déchainaient des tempêtes et il conservait en lui le chaos originel, indompté et tempétueux. Parmi les \Dormus, il était le seul à saisir le potentiel de ce qui lui avait été donné, et sa condition présente, dès le début, n'était pour lui qu'une coquille à briser. Il méprisait \Boromu, et il haïssait \Tot. \Tot, le quatrième, brûlait d'un blanc pur et aveuglant. Il avait immédiatement redouté son existence en tant qu'elle impliquait irrémédiablement sa propre disparition. Il avait fait de cette dialectique le cœur même de son existence et cette tension fût si forte qu'il finit par donner naissance à des \SC. Les œufs qu'il disséminait sur son passage, et dans lesquels couvaient son feu et une part de lui-même, apaisaient l'angoisse de sa certaine disparition. Pour \Ogo, dont les désirs fleurissaient inlassablement le long de murs qui ne demandaient plus qu'à être abattus, \Tot était la volonté en mouvement sous sa forme la plus pathétique et méprisable. 

%\Ogo supportait mal ses congénères, il pensait qu'il devait être l'unique être à posséder une volonté.

\section{L'\Extinction}

\subsection{La duperie d'\Ogo}

 \Ogo demandait sans relâche à \Mey davantage de pouvoir. \Mey, sur les conseils de \Cind, rejetait ses demandes et lui enjoignait la patience. Mais \Ogo brûlait d'un désir si grand qu'il vécut le don de volonté sans moyens de la réaliser comme une humiliation et grandit en lui une rancœur terrible.  
 
 
  Ses demandes laissées sans réponse lui apprirent peu à peu le moyen d'atteindre \Mey et de le contraindre à accepter de lui donner ce qu'il lui revenait. \Ogo alla à la rencontre d'\Esu, dont il devina facilement les faiblesses. \Ogo dit à \Esu que la reconnaissance de \Mey ne pouvait lui être définitivement témoignée que s'il se distinguait par la réalisation d'un exploit. Il flatta sa beauté et ses qualités et parvint à nourrir l'idéal de son égo blessé. Pour s'illustrer l'exploit devait être grand et il ne fallait pas lésiner sur les moyens. \Ogo lui demanda de lui indiquer le point du chaos le plus instable que sa merveilleuse sensibilité pouvait résoudre parmi l'immensité des courants de feu. \Ogo se proposa d'en attiser le mouvement et les flammes jusqu'à ce que cela nuise fortement à \Mey. Alors \Esu se présenterait, et grâce à sa grande influence sur les \SC, il apaiserait de manière orchestrale et à lui seul les tumultes et les douleurs, et \Mey ne pourrait que tenir en souveraine estime et fierté le plus grand des \Dormus. \Esu dédaignait \Ogo et il n'avait aucune confiance en cet être trop proche du chaos. Mais son désespoir était si complet qu'il se laissa convaincre, et indiqua à \Ogo l'endroit qu'il cherchait. Il s'y rendit, s'écoulant sans fin parmi les \SC. 
  Il repéra alors le point du chaos autour duquel des courants de flammes rouges enténébrés tourbillonnaient en une danse carnivore, s'aspirant et se rejetant sans cesse, comme dans une lutte à mort que le destin avait déserté,  dans un grands fracas et d'ondes de chocs qui faisaient vibrer et disloquaient les \SC qui s'écoulaient trop près de ces tempêtes en duel. \Ogo se joignit aux tourbillons, et tout en accélérant les mouvements de flammes en créa de nouveaux à l'intérieur des plus grands, il força les \SC aux alentours à suivre des trajectoires tortueuses et à amplifier la circulation des courants de flammes. Il les conduisait dans les tourbillons, les faisaient suivre les courbes vers le centre jusqu'à ce que les turbulences violemment les disloquent, et ainsi ce grand bûcher s'épaississait du désordre. La tempête finit par prendre une telle ampleur que \Mey se pâmait subitement de douleur, il fut à la fois subjugué par sa beauté, car rien encore dans cet univers ne fut aussi beau et unique, et terrassé en lui même par la douleur qu'elle lui causait. Alors \Esu finit par s'y rendre, mais lorsqu' il arriva la tempête était si violente et si forte qu'il ne put rien faire pour la calmer par ses propres moyens. La tempête de chaos se composaient de tant de tourbillons entrelacés, ses mouvements entrainaient tous les courants de flammes dans sa danse et elle ne cessait de s'accroitre, elle dévorait les \SC et alimentait son propre désordre en engloutissant l'ordre des \SC. Plus elle dévorait d'\SC et plus elle gagnait en puissance, elle enflait sans fin et elle soumettait l'espace même à ses propres tourments. \Esu se résigna et finit par demander de l'aide à \Boromu, mais des éternités entières se seraient écoulées avant qu'il puisse l'atteindre et la tempête aurait pris alors tellement d'ampleur qu'elle aurait aspiré \Boromu, le plus grands des \Dormus, pour en faire une flamme bleue de l'un de ses bras dévorants l'espace. \Boromu le savait et il comprit à la fois tous les évènements à l'origine de cette tempête mais également toutes les conséquences qu'elle aurait. Au lieu d'écouter les appels à l'aide d'\Esu, qui pathétique, maudissait \Ogo et s'apitoyait sur son propre sort. \Boromu réunissa tous les œufs de \Tot qu'il rencontrait et les engloutit. \Tot, aussitôt qu'il eut pris connaissance de l'existence de la tempête s'était enfui aux confins de l'univers et aucun \Dormus ne le revit jamais.      
   
   
  \subsection{La chute de \Mey}
  
  \Mey était étrangement hypnotisé par la tempête. Elle avait une forme si particulière, elle dansait dans son esprit, tumultueuse et enchanteresse, elle lui donnait à voir des couleurs et des motifs nouveaux dans des associations qu'il n'aurait jamais pu imaginer lui-même. C'était une source d'inspiration et la détruire lui était chose impossible car il n'avait jamais rien vu de si beau, et il l'appela \Auga. Mais \Auga à force de croitre engloutissait l'esprit de \Mey et menaçait de le faire tomber dans l'oubli. \Cind n'eut d'autre choix que d'intervenir: de toute sa force il souffla sur l'univers de \Mey. Le front de l'onde de choc dévalait l'espace sans faiblir, sa vitesse était si grande que les tumultes de flammes se pétrifiaient à son approche avant de se faire avaler, et sa puissance balaya l'univers tout entier, anéantissant tout mouvement et balayant toutes les flammes: la tempête se glaça, cristallisée dans sa forme et sa structure étrange et fractale, et le premier endroit du monde disparut aussitôt dans les ténèbres. L'être Unique survécut mais \Mey, malgré toute sa raison, en voulut tellement à \Cind, qu'il se scinda en deux êtres distincts. \Mey, en proie à la plus grande des tristesses et à la plus noire des mélancolies, était en train de disparaitre et il ne put survivre à cette séparation. Dans un dernier effort il sortit l'univers de ses pensées. \Mey, l'être unique, las de son éternité, et sans mot dire à \Cind, quitta à jamais la solitude de son existence, et une partie de lui se projeta dans son univers, sa dernière idée, sous la forme libre et insoucieuse d'un météore.
  
  
 % Meydra en danger. Cindara fera tout pour l’empecher en anéantissant les flammes, en refroidissant l’univers et en essayant de tuer les Omu [Donner un nom a cet evenement] Certains survivront (voir au dessus). Meydra malgré sa propre mise en danger de disparition en voudra terriblement à Cindara pour son geste, et l’être unique se séparera en 2 etres distincts. Finalement Meydra, apres plusieurs soubresaults et un dernier coup de création de génie ( a trouver) finira par disparaître en faisant part à Cindara de ses intentions quant à cet univers (quelles étaient elles ? Peut etre de pour une fois créer quelque chose, un monde, qu’il ne chercherait pas a controler mais qui ferait par lui meme des choses plus merveilleuses que lui, en préparant juste la terre et y laisser pousser les graines. Ou alors décider de ne faire que créer et ne rien détruire soi meme, interdiction de tuer quoi que ce soit, juste ajouter ne jamais rien enlever) et seul restera Cindara, qui fera tout pour continuer l’œuvre de Meydra en essayant de batir un univers harmonieux en l’honneur de Meydra qui était toujours dévoré par ses créations. 
  
%\chapter{L'âge des étoiles}

\subsection{\Cind}

\Cind se retrouva seul et dans un état d'affliction qu'il n'avait jamais connu. \Mey lui avait fait par de sa volonté de bâtir une idée qui se développerait d'elle même, sous sa propre musique, où eux, l'être Unique, n'aurait que à coeur de la laisser évoluer. \Cind avait fait le serment à \Mey de ne jamais détruire quoi que ce soit dans l'univers. \Cind appela cet univers \dreyma et exauça les derniers souhaits de sa part disparue. \Cind n'avait pas la créativité de \Mey il s'imprégna alors de sa propre philosophie et décida de faire ré apparaitre la lumière. Il créa \carac, un être divin sans paroles et au but unique: faire reculer les ténèbres et ramener la lumière. \carac apparut dans \dreyma sous la forme d'un spectre, \Cind lui donna un immense marteau d'un noir sans reflet car il remarqua que seuls les oeufs de \Tot avaient survécu au souffle et qu'en eux couvaient la lumière du chaos originel. \Boromu, dans sa grande sagesse, avaient protégé les oeufs de ses puissantes flammes bleues, il avait absorbé le souffle de \Cind.  \carac se mit à la tâche, il alla d'oeuf en oeuf pour en briser la coquille et libérer l'éclat du feu à présent inerte et la lumière se remis à arpenter les espaces infins. Il sculpta leur éclat, en attisa les flammes et leur redonna leur éclat d'antan en ajustant l'équilibre des couleurs comme les fleurs d'un jardin d'obscurité. Il les déplaça et les arrangea pour éclairer l'univers de la façon la plus harmonieuse selon n'importe quel point de vue. Les couleurs des flammes, bleues et rouges, vertes et blanches, jaunes et cyan s'enrobaient les unes dans les autres et de nouvelles couleurs naquirent. \carac, en hommage à \Boromu, brisa la coquille de dix oeufs pour en former un seul, d'un bleu puissant et calme, et forma l'étoile la plus grande que le monde ait connue. Il l'alluma d'un coup de marteau bien ajusté et elle brille encore. En silence, son lourd marteau enrobé de flammes du chaos originel frappait avec une régularité infatigable les oeufs aux quatres coins du monde. Le rythme régulier de ses coups faisait vibrer la structure de l'espace et du marteau de \carac naquit le temps, première horloge connue du monde, dont les échos résonnaient dans l'immensité du vide. Le météore issu de \Mey, Dristt, dansait parmi les étoiles qui s'allumaient peu à peu. Elle décrivait des courbes gracieuses et voguait avec grâce et légèreté dans l'espace, comme un être qui en renonçant à une partie de lui meme était devenu enfin libre. En passant dans leur voisinage ses trajectoires s'illuminaient d'un majestueuse trainée scintillante reflétant des couleurs nouvelles et resplendissantes.


\subsection{\Ogo}

Les œufs de \Tot survécurent à l'\Extinction et parmi l'un d'eux \Ogo avait réussi à se cacher.

\backmatter
% bibliography, glossary and index would go here.

\printglossary[title=Noms propres, toctitle=Lieux et Entités]

\end{document}