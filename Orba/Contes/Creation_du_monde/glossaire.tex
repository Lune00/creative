%Template:
%\newEntity{Nom commande pour appeler dans le texte}{Entree glossaire}{Apparait dans le texte}{Description}	

%Faire differents glossaires


%Etre unique:

\newEntity{Cind}{Cindara}{Cindara}{: une des deux faces, avec \Mey, de l'être suprême}

\newEntity{Mey}{Meydra}{Meydra}{: une des deux faces, avec \Cind, de l'être suprême. Il est la partie la plus puissante de l'être mais aussi la plus fragile}

\newEntity{Drisst}{Drisst}{Drisst}{: projection et réduction de \Mey dans le \reel sous la forme d'un météore signant la fin de son existence}

% Omus et Dormus:

\newEntity{SC}{Omus}{Ömus}{: les serpents cosmiques sont les premiers êtres de l'univers. Ils sont imaginés par \Cind et \Mey leur donne vie. Durant l'Âge du chaos ils assurent la persistance de l'idée-monde}

\newEntity{Dormus}{Dormus}{Dörmus}{: à l'origine des \SC. \Mey leur donne chacun une volonté propre. Ils sont au nombre de quatre: \Boromu, \Esu, \Tot et \Ogo}

\newEntity{Boromu}{Boromu}{Boromu}{: un des quatre \Dormus. Immense, le plus puissant d'entre eux, il pacifie le chaos sans relâche. Pressentant l'\Extinction il protège les œufs de \Tot avant de disparaître}

\newEntity{Esu}{Esu}{Esu}{: un des quatre \Dormus. Le plus parfait et le plus malheureux des \Dormus. En manque de reconnaissance de \Mey il est dupé par \Ogo, et lui révèle le point du chaos le plus instable}

\newEntity{Tot}{Tot}{Töt}{: un des quatre \Dormus. Ses œufs ont permis à \Carac de créer les étoiles }

\newEntity{Ogo}{Ogo}{Ogo}{: un des quatre \Dormus. Il donna naissance à \Auga et fût le seul \Dormus à échapper à l'\Extinction}


%Evenements:

\newEntity{Extinction}{Ex}{Extinction}{: marque la fin de l'ère du chaos et des \SC. En réponse à la duperie d'\Ogo, \Cind souffla toutes les flammes et ramena l'univers dans les ténèbres}

%Lieux:

\newEntity{Auga}{temp}{Aùga}{: tempête créée par \Ogo à partir du point du chaos le plus instable de l'univers connu par \Esu. Elle fit un trou dans l'univers ouvert sur la dimension de l'être suprême}

\newEntity{Dreyma}{dreyma}{Dreyma}{: nom donné par \Cind à l'univers de \Mey après l'\Extinction}

\newEntity{Rdr}{Rd}{Royaume des rêves}{: crée par \Onodine}

\newEntity{Orba}{Orba}{Orba}{: la Terre, ou le monde. Sculptée par les \Ea}

%ETOILES

\newEntity{Boromil}{Boromil}{Boromil}{: étoile immense et bleue. Elle est spécialement créee par \Carac en l'honneur de \Boromu. Elle donne la mesure du temps aux autres étoiles}

\newEntity{Naos}{Naos}{Naos}{: grande étoile d'une lumière d'or. Elle signifie \textit{navire}. Elle est à la fois la mère et le berceau des \Ea}

%Dieux:

\newEntity{Ea}{Ea}{Eà}{: les êtres imaginants, surgissent de l'étoile \Naos lorsqu'elle finit de consumer le dernier fragment de \Drisst. Ils sont au nombre de 3: \Oros, \Nio et \Fercor}


%\newEntity{}{}{}{: }
\newEntity{Oros}{Oros}{Oros}{: aussi Aros, Aroë }
\newEntity{Nio}{Nio}{Nío}{: aussi Merildaar }
\newEntity{Fercor}{Fercor}{Fercor}{: }


%reve arpenteur voyageur drogue

\newEntity{Shuru}{Shuru}{Shuru}{: appelée par \Ogo, passe dans l'univers par la déchirure  d'\Auga. Libère \Ogo et après la réception d'un fragment de \Drisst il rebatit les imaginaires de \Mey appelé \Rdr}

\newEntity{Onodine}{Onodine}{Onodine}{: anciennement \Shuru, crée et gouverne le \Rdr. Il se nomme "L'\Ea des rêves", avant que le \Rdr ne soit renversé lors de la guerre du Rêve}

\newEntity{Carac}{carac}{Caracor}{: crée par \Cind, il est le dieu des étoiles et sculpteur du chaos}

%Autres:


\newEntity{reel}{reel}{Réel}{: espace de l'imaginaire déserté  par l'imagination de \Mey, où les images sont figées, et existent en et par elles-même. La causalité y règne en tant que le rapport de cause à effet entre évènements ne peut plus trouver sa source en dehors de ses propres limites, mais seulement dans le rapport entre les images elles-même; au contraire des espaces imaginaires, où le lien de causalité, bien que toujours existant, ne peut pas être nécessairement invoqué car des évènements peuvent trouver leur cause dans un autre espace, et que la succession des évènements est en outre issue de l'angoisse de l'origine de \Mey, insondable, et les causes bien souvent inaccessibles. \Shuru méprise le réel car il est une réduction de l'imaginaire, \Ogo au contraire y voit là le seul espace de liberté. \Ogo, en un sens, est l'irruption, le point de cristallisation du réel au sein de l'imaginaire de \Mey, qui mettra fin à la fois à \Mey et à son tourment, et détruira pour toujours la possibilité de répondre à la question de savoir s'il pré-existait quelque-chose à l'Unique et ce qui a nourri son imagination}